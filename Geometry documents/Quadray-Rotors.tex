\documentclass[11pt,a4paper]{article}
\usepackage[utf8]{inputenc}
\usepackage[T1]{fontenc}
\usepackage{amsmath,amssymb,amsthm}
\usepackage{geometry}
\usepackage{hyperref}
\usepackage{graphicx}
\usepackage{booktabs}
\usepackage{enumitem}
\usepackage{xcolor}
\usepackage{parskip}  % Adds vertical space between paragraphs
\usepackage{tcolorbox}  % For framed boxes
\usepackage{endnotes}   % For endnotes instead of footnotes

\geometry{margin=1in}

\hypersetup{
    colorlinks=true,
    linkcolor=blue,
    urlcolor=blue,
    citecolor=blue
}

\newtheorem{conjecture}{Conjecture}
\newtheorem{definition}{Definition}
\newtheorem{observation}{Observation}
\newtheorem{theorem}{Theorem}
\newtheorem{proposition}{Proposition}

\title{Spread-Quadray Rotors -- v2.0 (Feb 2026)\\
\large A Tetrahedral Alternative to Quaternions\\
for Gimbal-Lock-Free Rotation Representation}

\author{Andrew Thomson\\
\small Open Building / ARTexplorer Project\\
\small \href{mailto:andy@openbuilding.ca}{andy@openbuilding.ca}}

\date{February 2026}

\begin{document}

\maketitle

\begin{abstract}
This document proposes a novel approach to 3D rotation representation: \textbf{Spread-Quadray Rotors}. By combining full 4D Quadray coordinates (without zero-sum constraint), Rational Trigonometry (spread/cross instead of sin/cos), half-angle algebra, and explicit Janus polarity, we arrive at a rotation system that is gimbal-lock free, algebraically exact for many useful rotations, and geometrically native to tetrahedral structures.

The \textbf{Hairy Ball Theorem} (Brouwer, 1912) establishes that any continuous tangent vector field on a sphere must have at least one singularity---you cannot ``comb a hairy ball flat without creating a cowlick.'' This topological obstruction is why 3-parameter rotation representations (Euler angles, zero-sum Quadray) inevitably suffer gimbal lock. Quaternions escape this by lifting to the 4-dimensional hypersphere $S^3$; we propose an alternative lift to $\mathbb{R}^4 \times \mathbb{Z}_2$ via tetrahedral coordinates, yielding a fundamentally different (and arguably easier to understand) algebraic structure that preserves exact rational values throughout rotation calculations.

This is not merely ``quaternions in different clothing'' but a genuinely distinct representation arising from tetrahedral geometry and Wildberger's Rational Trigonometry.
\end{abstract}

\tableofcontents
\newpage

%==============================================================================
\section{Introduction}
%==============================================================================

\subsection{The Problem: Gimbal Lock}

Gimbal lock occurs when two rotation axes align, causing a loss of one degree of freedom. In Euler angle representation (pitch, yaw, roll), this happens at $\pm 90°$ pitch---the system loses the ability to distinguish yaw from roll.

\textbf{Physical analogy:} A camera gimbal with three nested rings. When the middle ring rotates $90°$, the inner and outer rings become parallel---you can no longer independently control two of the three rotations.

\textbf{Why it matters:}
\begin{itemize}
    \item \textbf{Animation:} Interpolating between orientations near gimbal lock produces erratic motion
    \item \textbf{Robotics:} Control systems can become unstable
    \item \textbf{Aerospace:} Apollo 11's guidance computer nearly encountered gimbal lock
    \item \textbf{3D Graphics:} SLERP requires quaternions to avoid interpolation artifacts
\end{itemize}

\subsection{The Standard Solution: Quaternions}

Quaternions avoid gimbal lock by using 4 parameters constrained to a 3-sphere ($S^3$):
\begin{equation}
q = q_0 + q_1 i + q_2 j + q_3 k \quad \text{where} \quad q_0^2 + q_1^2 + q_2^2 + q_3^2 = 1
\end{equation}

This works because $S^3$ is \emph{simply connected}---any two orientations can be connected by a continuous path without passing through a singularity.

\textbf{But quaternions have limitations:}
\begin{itemize}
    \item Require transcendental functions (sin, cos) for most angle values
    \item The unit norm constraint requires renormalization after composition
    \item The double-cover ($q$ and $-q$ represent the same rotation) is implicit, not explicit
\end{itemize}

\subsection{Our Proposal: Spread-Quadray Rotors}

We propose an alternative 4-parameter rotation representation based on:
\begin{enumerate}
    \item \textbf{Full 4D Quadray coordinates}---without zero-sum constraint
    \item \textbf{Spread/Cross measures}---from Wildberger's Rational Trigonometry
    \item \textbf{Half-angle algebra}---$\sin^2(\theta/2) = (1 - \cos\theta)/2$ for rotor construction
    \item \textbf{Janus polarity}---explicit discrete state for double-cover
\end{enumerate}

\textbf{Research direction:} Weierstrass parametrization ($t = \tan(\theta/2)$) offers fully algebraic circle points for rational $t$. A future enhancement could construct rotors directly from Weierstrass parameters, eliminating all $\sqrt{}$ operations for rational inputs. See Section~\ref{sec:weierstrass} for the mathematical foundation.

%==============================================================================
\section{Topological Foundations}
%==============================================================================

\subsection{The Hairy Ball Theorem}

\begin{theorem}[Hairy Ball Theorem --- Brouwer, 1912]
There is no nonvanishing continuous tangent vector field on even-dimensional $n$-spheres. In particular, on $S^2$, any continuous tangent vector field must have at least one point where the vector vanishes.
\end{theorem}

\textbf{Colloquial version:} ``You cannot comb a hairy ball flat without creating at least one cowlick.''

\textbf{Implications:}
\begin{itemize}
    \item There is always at least one point on Earth with zero wind (a cyclone eye or calm spot)
    \item You cannot texture-map a sphere without at least one singularity
    \item \textbf{Gimbal lock is unavoidable with 3-parameter rotation systems}
\end{itemize}

\begin{tcolorbox}[colback=blue!5, colframe=blue!40!black, title=\textbf{Escaping the Hairy Ball: Two Approaches}]
The cowlick exists because hair must lie flat \emph{on} the 2D spherical surface---there must be at least one point where no consistent direction can be assigned. How do we escape?

\textbf{Quaternions} escape by moving to $S^3$, a higher-dimensional sphere where combing \emph{is} possible. Think of it as finding a better-shaped surface---one without the topological obstruction.

\textbf{Quadray Rotors} escape differently: we use four tetrahedral directions instead of three orthogonal ones. Picture a tetrahedron at the origin, with axes pointing to its four vertices. While these four vertices are non-coplanar in 3D space, the key insight is that we store \emph{all four coordinates} rather than constraining to three. This creates a 4-parameter representation ($\mathbb{R}^4$) that avoids the topological trap---we're not confined to a 2D surface that must be combed, so there is no cowlick to create.

In the ARTexplorer demo, the four colored rings (W, X, Y, Z) represent rotation \emph{about} each tetrahedral axis---gumball-style handles that let you spin the orientation around any of the four basis directions. The smooth motion, even where Euler angles would fail, demonstrates the absence of singularities in tetrahedral 4D space.
\end{tcolorbox}

\begin{tcolorbox}[colback=yellow!5, colframe=yellow!50!black, title=\textbf{Did You Know? Fuller's Precession and Gimbal Lock}]
R.\ Buckminster Fuller described \textbf{precession} as the 90° offset between applied force and resulting motion---the way a spinning gyroscope ``walks'' perpendicular to a push. Fuller saw precession as fundamental to how Universe operates: effects manifest at right angles to causes.

At gimbal lock, two rotation axes \emph{align}---they become parallel rather than maintaining their angular offset. The ``precessional pathway'' is blocked; there's no perpendicular direction left to precess into.

With Quadray's $109.47°$ tetrahedral angles, the basis vectors are \emph{more obtuse} than orthogonal $90°$. They never align through any continuous motion in 4D space. The tetrahedron's geometry is \textbf{precessionally complete}---it maintains angular ``room to maneuver'' that Cartesian systems lose at their singularities. Where Euler angles hit a precessional speedbump, tetrahedral rotations glide through on more obtuse angular transitions.
\end{tcolorbox}

\subsection{Why 3 Parameters Are Not Enough}

The space of 3D rotations is $SO(3)$---a 3-dimensional manifold that is \emph{not simply connected}. Topologically, $SO(3)$ is equivalent to $\mathbb{RP}^3$ (real projective 3-space), which has a ``twist'' that prevents global singularity-free parameterization with only 3 numbers.

\begin{observation}[Topological Obstruction]
You cannot parameterize $SO(3)$ with 3 continuous parameters without creating at least one singularity. This is a direct consequence of the Hairy Ball Theorem applied to rotation space.
\end{observation}

\subsection{The Lift to 4D}

The solution is to \textbf{lift} from $SO(3)$ to a higher-dimensional space:

\begin{table}[h]
\centering
\begin{tabular}{cccc}
\toprule
\textbf{Space} & \textbf{Dimension} & \textbf{Topology} & \textbf{Singularities}\\
\midrule
$SO(3)$ & 3 & $\mathbb{RP}^3$ (twisted) & Unavoidable\\
$S^3$ (Quaternions) & 3-sphere in $\mathbb{R}^4$ & Simply connected & None\\
$\mathbb{R}^4$ (Full Quadray) & 4D Euclidean & Simply connected & None\\
\bottomrule
\end{tabular}
\caption{Topological comparison of rotation representation spaces}
\end{table}

Both $S^3$ (quaternions) and $\mathbb{R}^4$ (full Quadray) provide this lift, but with different geometric structures.

%==============================================================================
\section{Quadray vs Quaternion Topology}
%==============================================================================

\subsection{Comparison of Representations}

\begin{table}[h]
\centering
\small
\begin{tabular}{lccccc}
\toprule
\textbf{Representation} & \textbf{Scalars} & \textbf{Constraint} & \textbf{Manifold} & \textbf{Lift} & \textbf{Gimbal Lock}\\
\midrule
Euler angles & 3 & None & $SO(3)$ chart & None & \textbf{Yes}\\
Quadray (zero-sum) & $4 \to 3$ & $w+x+y+z = k$ & $SO(3)$ chart & None & \textbf{Yes}\\
\textbf{Quadray (full 4D)} & 4 & None & $\mathbb{R}^4$ & Implicit $\mathbb{R}^4$ & \textbf{No}$^*$\\
Quaternions & 4 & $\|q\| = 1$ & $S^3$ & Explicit Spin(3) & \textbf{No}\\
\bottomrule
\end{tabular}
\caption{Comparison of rotation representations. $^*$Gimbal lock avoided provided the 4 scalars parameterize orientation directly and are not reduced to a 3-parameter $SO(3)$ chart.}
\end{table}

\subsection{The Zero-Sum Trap}

\begin{observation}[Critical Distinction]
When we enforce $W + X + Y + Z = \text{constant}$, we \textbf{project} 4D Quadray back to 3D. This projection reintroduces the $SO(3)$ topology and its inherent singularities.
\end{observation}

\begin{equation}
\text{Full Quadray } (\mathbb{R}^4) \xrightarrow{\text{zero-sum constraint}} \text{Projected Quadray } (\mathbb{R}^3) \cong SO(3)
\end{equation}

\textbf{The zero-sum constraint is a projection, not a necessity.} To avoid gimbal lock, we must work in the full 4D space.

\subsection{Why Full Quadray Differs from Quaternions}

\begin{table}[h]
\centering
\begin{tabular}{lll}
\toprule
\textbf{Aspect} & \textbf{Quaternions} & \textbf{Full Quadray}\\
\midrule
Basis geometry & Orthogonal ($90°$ between $i, j, k$) & Tetrahedral ($109.47°$ between $W, X, Y, Z$)\\
Constraint & Unit norm (forces onto $S^3$) & None (free in $\mathbb{R}^4$)\\
Double-cover & Implicit ($q \equiv -q$) & Explicit (Janus polarity bit)\\
Composition & Hamilton product & Tetrahedral rotation matrices\\
Interpolation & SLERP on $S^3$ & Linear in $\mathbb{R}^4$ (simpler!)\\
\bottomrule
\end{tabular}
\caption{Structural differences between quaternions and full Quadray}
\end{table}

%==============================================================================
\section{Spread-Quadray Rotors: Definition}
%==============================================================================

\subsection{The Core Concept}

Instead of using angle $\theta$ with sin/cos, we use:
\begin{itemize}
    \item \textbf{Spread} $s = \sin^2(\theta)$---a rational value for many useful angles
    \item \textbf{Cross} $c = \cos^2(\theta) = 1 - s$---the complementary measure
    \item \textbf{Weierstrass parameter} $t$---generates exact rational sin/cos values
\end{itemize}

\begin{definition}[Spread-Quadray Rotor]
A \textbf{Spread-Quadray Rotor} $R$ is defined as:
\begin{equation}
R = (W, X, Y, Z, \pm) \in \mathbb{R}^4 \times \mathbb{Z}_2
\end{equation}
where:
\begin{itemize}
    \item $(W, X, Y, Z)$ are four independent scalars (no zero-sum constraint)
    \item $\pm$ is the \textbf{Janus polarity} (discrete: positive or negative dimensional space)
\end{itemize}
\end{definition}

\subsection{Rotor Parameters from Spread}

Given spread $s$ and Weierstrass parameter $t$:
\begin{align}
t &= \sqrt{\frac{s}{1-s}} \quad \text{(parameter from spread)}\\
\cos(\theta) &= \frac{1 - t^2}{1 + t^2} \quad \text{(algebraic, no transcendentals!)}\\
\sin(\theta) &= \frac{2t}{1 + t^2} \quad \text{(algebraic, no transcendentals!)}
\end{align}

\subsection{Why ``Rotor'' Not ``Quaternion''}

We deliberately avoid calling these ``Quadray quaternions'' because:
\begin{enumerate}
    \item \textbf{Different algebra:} Quaternions use Hamilton multiplication ($ij = k$, etc.). Quadray rotors use tetrahedral rotation matrices.
    \item \textbf{Different constraint:} Quaternions require $\|q\| = 1$. Quadray rotors have no norm constraint.
    \item \textbf{Different topology:} Quaternions live on $S^3$. Quadray rotors live in $\mathbb{R}^4 \times \mathbb{Z}_2$.
    \item \textbf{Different exactness:} Quaternions require transcendentals for most angles. Quadray rotors can be exact rational for many useful rotations.
\end{enumerate}

%==============================================================================
\section{RT-Pure Rotation Mathematics}
%==============================================================================

\subsection{Spread and Cross}

From Wildberger's Rational Trigonometry:
\begin{align}
\text{Spread:} \quad s &= \sin^2(\theta) \quad \text{(measures ``perpendicularity'': 0 = parallel, 1 = perpendicular)}\\
\text{Cross:} \quad c &= \cos^2(\theta) \quad \text{(complementary measure)}\\
\text{Identity:} \quad s + c &= 1
\end{align}

\begin{observation}[Rational Spreads]
Spread is often a \textbf{rational number} even when $\sin(\theta)$ is irrational.
\end{observation}

\begin{table}[h]
\centering
\begin{tabular}{ccccc}
\toprule
\textbf{Angle} $\theta$ & $\sin(\theta)$ & $\cos(\theta)$ & \textbf{Spread} $s$ & \textbf{Cross} $c$\\
\midrule
$0°$ & 0 & 1 & 0 & 1\\
$30°$ & $1/2$ & $\sqrt{3}/2$ & \textbf{1/4} & \textbf{3/4}\\
$45°$ & $\sqrt{2}/2$ & $\sqrt{2}/2$ & \textbf{1/2} & \textbf{1/2}\\
$60°$ & $\sqrt{3}/2$ & $1/2$ & \textbf{3/4} & \textbf{1/4}\\
$90°$ & 1 & 0 & \textbf{1} & \textbf{0}\\
\bottomrule
\end{tabular}
\caption{Spread and cross values for common angles---note that all spreads are exact rationals}
\end{table}

\subsection{Weierstrass Parametrization}
\label{sec:weierstrass}

The Weierstrass substitution provides \textbf{algebraic} circle points:
\begin{align}
t &= \tan(\theta/2) \quad \text{(the parameter)}\\
\cos(\theta) &= \frac{1 - t^2}{1 + t^2}\\
\sin(\theta) &= \frac{2t}{1 + t^2}
\end{align}

\begin{theorem}[RT-Pure Benefit]
For any \textbf{rational} $t$, both $\cos(\theta)$ and $\sin(\theta)$ are \textbf{exact rational values}.
\end{theorem}

This is implemented in ARTexplorer's \texttt{rt-math.js}:
\begin{verbatim}
RT.circleParam = t => {
  const tSquared = t * t;
  const denominator = 1 + tSquared;
  return {
    x: (1 - tSquared) / denominator,  // cos(theta) -- algebraic!
    y: (2 * t) / denominator,         // sin(theta) -- algebraic!
  };
};
\end{verbatim}

\subsection{From Spread to Weierstrass Parameter}

Given spread $s$, find parameter $t$:
\begin{align}
s &= \sin^2(\theta) = \left[\frac{2t}{1+t^2}\right]^2 = \frac{4t^2}{(1+t^2)^2}\\
\text{Solving:} \quad s(1 + t^2)^2 &= 4t^2\\
s \cdot t^4 + (2s - 4)t^2 + s &= 0
\end{align}

Using the quadratic formula with $u = t^2$:
\begin{align}
u &= \frac{4 - 2s \pm \sqrt{16 - 16s}}{2s} = \frac{2 - s \pm 2\sqrt{1-s}}{s}\\
t &= \sqrt{u} = \sqrt{\frac{2 - s + 2\sqrt{1-s}}{s}} \quad \text{(taking positive root)}
\end{align}

For exact rational spreads, this often simplifies beautifully.

\subsection{Research Direction: Weierstrass-Based Rotor Construction}

The current implementation uses half-angle identities to construct rotors from spread:
\begin{verbatim}
// Current approach (half-angle algebra)
const cosTheta = polarity * Math.sqrt(c);    // c = 1 - s (cross)
const spreadHalf = (1 - cosTheta) / 2;       // sin²(θ/2)
const crossHalf = (1 + cosTheta) / 2;        // cos²(θ/2)
const sinHalf = Math.sqrt(spreadHalf);
const cosHalf = Math.sqrt(crossHalf);
\end{verbatim}

A future enhancement could construct rotors directly from Weierstrass parameter $t$:
\begin{verbatim}
// Research direction (Weierstrass-based)
// Input: t = tan(θ/2) instead of spread s
const tSq = t * t;
const denom = 1 + tSq;
const cosTheta = (1 - tSq) / denom;    // Algebraic!
const sinTheta = (2 * t) / denom;      // Algebraic!

// For rotor: need cos(θ/2), sin(θ/2)
// Using half-angle of half-angle: t' = tan(θ/4)
// This remains algebraic for rational t
\end{verbatim}

\textbf{Benefit:} For rational $t$, all intermediate values remain exact rationals---no $\sqrt{}$ operations required. This would complete the RT-pure vision for the subset of rotations expressible with rational Weierstrass parameters.

%==============================================================================
\section{The Tetrahedral Rotation Matrix}
%==============================================================================

\subsection{Tom Ace's Quadray Rotation Formula}

Rotation about a Quadray axis uses coefficients $F$, $G$, $H$:
\begin{align}
F &= \frac{2\cos(\theta) + 1}{3}\\
G &= \frac{2\cos(\theta - 120°) + 1}{3}\\
H &= \frac{2\cos(\theta + 120°) + 1}{3}
\end{align}

\subsection{RT-Pure Form (Using Spread)}

\begin{verbatim}
function rotationCoeffsFromSpread(s) {
  const c = 1 - s;              // cross = cos^2(theta)
  const cosTheta = Math.sqrt(c);  // Deferred sqrt until needed
  const sinTheta = Math.sqrt(s);

  const cos120 = -0.5;           // -1/2 exactly
  const sin120 = Math.sqrt(0.75); // sqrt(3/4)

  const F = (2 * cosTheta + 1) / 3;
  const G = (2 * (cosTheta * cos120 + sinTheta * sin120) + 1) / 3;
  const H = (2 * (cosTheta * cos120 - sinTheta * sin120) + 1) / 3;

  return { F, G, H };
}
\end{verbatim}

\subsection{The 4×4 Rotation Matrix}

Rotation about the $W$-axis by spread $s$:
\begin{equation}
R = \begin{pmatrix}
1 & 0 & 0 & 0\\
0 & F & H & G\\
0 & G & F & H\\
0 & H & G & F
\end{pmatrix}
\end{equation}

\begin{observation}[Circulant Structure]
Note the \textbf{circulant structure} of the $3 \times 3$ submatrix---this reflects the tetrahedral symmetry where all non-axis coordinates are treated equivalently.
\end{observation}

%==============================================================================
\section{Exact Rational Rotations}
%==============================================================================

\subsection{The Gold Standard}

For certain angles, spread and cross are \textbf{exact rationals}, enabling algebraically exact rotation:

\begin{table}[h]
\centering
\begin{tabular}{ccccccl}
\toprule
\textbf{Rotation} & \textbf{Spread} $s$ & \textbf{Cross} $c$ & $F$ & $G$ & $H$ & \textbf{Notes}\\
\midrule
$0°$ & 0 & 1 & 1 & 0 & 0 & Identity\\
$30°$ & $1/4$ & $3/4$ & $(\sqrt{3}+1)/3$ & ... & ... & ---\\
$45°$ & $1/2$ & $1/2$ & $(\sqrt{2}+1)/3$ & ... & ... & Exact spread!\\
$60°$ & $3/4$ & $1/4$ & $2/3$ & $(\sqrt{3}+1)/6$ & $(1-\sqrt{3})/6$ & Tetrahedral\\
$90°$ & 1 & 0 & $1/3$ & $(\sqrt{3}+1)/3$ & $(1-\sqrt{3})/3$ & Quarter turn\\
$120°$ & $3/4$ & $1/4$ & 0 & $1/3$ & $1/3$ & \textbf{All rational!}\\
$180°$ & 0 & 1 & $-1/3$ & $2/3$ & $2/3$ & \textbf{All rational!}\\
\bottomrule
\end{tabular}
\caption{Rotation coefficients for common angles}
\end{table}

\subsection{Special Case: 120° Rotation (Tetrahedral Symmetry)}

At $120°$, the rotation coefficients become beautifully simple:
\begin{equation}
F = 0, \quad G = \frac{1}{3}, \quad H = \frac{1}{3}
\end{equation}

The rotation matrix becomes:
\begin{equation}
R_{120°} = \begin{pmatrix}
1 & 0 & 0 & 0\\
0 & 0 & 1/3 & 1/3\\
0 & 1/3 & 0 & 1/3\\
0 & 1/3 & 1/3 & 0
\end{pmatrix}
\end{equation}

This is a \textbf{cyclic permutation} of the non-axis coordinates---pure tetrahedral symmetry!

\subsection{Special Case: 180° Rotation (Janus Inversion)}

At $180°$, we get another exact rational matrix:
\begin{equation}
F = -\frac{1}{3}, \quad G = \frac{2}{3}, \quad H = \frac{2}{3}
\end{equation}

This represents the \textbf{Janus Point passage}---inversion through the origin from $4D^+$ to $4D^-$ space.

%==============================================================================
\section{The Janus Polarity Extension}
%==============================================================================

\subsection{Beyond Continuous Parameters}

Quaternions have an implicit double-cover: both $q$ and $-q$ represent the same rotation. This is mathematically elegant but can cause confusion (interpolation taking the ``long way around'').

Quadray rotors make this \textbf{explicit} with a discrete polarity bit:
\begin{equation}
\text{Full specification:} \quad (W, X, Y, Z, \pm)
\end{equation}

\begin{table}[h]
\centering
\begin{tabular}{ccc}
\toprule
\textbf{Polarity} & \textbf{Dimensional Space} & \textbf{Interpretation}\\
\midrule
$+$ & $4D^+$ & ``Positive'' dimensional space (outward from origin)\\
$-$ & $4D^-$ & ``Negative'' dimensional space (through origin, inverted)\\
\bottomrule
\end{tabular}
\caption{Janus polarity states}
\end{table}

The Janus Point (origin) is the \textbf{transition} between these spaces---passing through it flips the polarity.

\subsection{Advantages of Explicit Polarity}

\begin{enumerate}
    \item \textbf{Unambiguous interpolation:} Always know which ``sheet'' you're on
    \item \textbf{Cleaner animation:} No sudden flips when $q$ crosses $-q$ boundary
    \item \textbf{Physical meaning:} Connects to Fuller's ``inside-outing'' and cosmological Janus Point theories
    \item \textbf{Explicit topology:} The $\mathbb{Z}_2$ factor makes the double-cover visible
\end{enumerate}

%==============================================================================
\section{Implementation Roadmap}
%==============================================================================

\subsection{Current State (rt-math.js)}

ARTexplorer already implements key building blocks:
\begin{itemize}
    \item \texttt{RT.spread(v1, v2)}---Spread between vectors
    \item \texttt{RT.circleParam(t)}---Weierstrass parametrization
    \item \texttt{RT.spreadToParam(s)}---Convert spread to Weierstrass $t$
    \item \texttt{Quadray.toCartesian()}---Conversion (but uses zero-sum)
    \item \texttt{Quadray.fromCartesian()}---Reverse conversion
\end{itemize}

\subsection{Phase 1: Full 4D Quadray Storage}

\begin{verbatim}
class QuadrayRotor {
  constructor(w, x, y, z, polarity = '+') {
    this.w = w;
    this.x = x;
    this.y = y;
    this.z = z;
    this.polarity = polarity;  // '+' or '-'
  }
}
\end{verbatim}

\subsection{Phase 2: RT-Pure Rotation}

\begin{verbatim}
QuadrayRotor.prototype.rotateAboutW = function(spread) {
  const { F, G, H } = RT.rotationCoeffsFromSpread(spread);
  return new QuadrayRotor(
    this.w,
    F * this.x + H * this.y + G * this.z,
    G * this.x + F * this.y + H * this.z,
    H * this.x + G * this.y + F * this.z,
    this.polarity
  );
};
\end{verbatim}

\subsection{Phase 3: Composition and Interpolation}

\begin{verbatim}
// Linear interpolation in R^4 (simpler than SLERP!)
QuadrayRotor.lerp = function(r1, r2, t) {
  return new QuadrayRotor(
    r1.w + t * (r2.w - r1.w),
    r1.x + t * (r2.x - r1.x),
    r1.y + t * (r2.y - r1.y),
    r1.z + t * (r2.z - r1.z),
    t < 0.5 ? r1.polarity : r2.polarity
  );
};
\end{verbatim}

%==============================================================================
\section{Summary Comparison}
%==============================================================================

\begin{table}[h]
\centering
\begin{tabular}{lccc}
\toprule
\textbf{Feature} & \textbf{Euler Angles} & \textbf{Quaternions} & \textbf{Quadray Rotors}\\
\midrule
Parameters & 3 & 4 (constrained) & 4 + 1 discrete\\
Manifold & $SO(3)$ & $S^3$ & $\mathbb{R}^4 \times \mathbb{Z}_2$\\
Gimbal lock & Yes & No & No\\
Exact rationals & Rarely & Rarely & Often\\
Transcendentals & Always & Usually & Deferred\\
Interpolation & Problematic & SLERP & Linear\\
Double-cover & N/A & Implicit & Explicit\\
Native geometry & Orthogonal & Orthogonal & \textbf{Tetrahedral}\\
\bottomrule
\end{tabular}
\caption{Summary comparison of rotation representations}
\end{table}

%==============================================================================
\section{Native Quadray Rotation: The F,G,H Coefficients}
%==============================================================================

\subsection{Tom Ace's Tetrahedral Rotation Formula}

Tom Ace derived rotation coefficients that operate \emph{directly} on Quadray coordinates without conversion to Cartesian space. For rotation by angle $\theta$ about one of the four basis axes:

\begin{align}
F &= \frac{2\cos(\theta) + 1}{3}\\
G &= \frac{2\cos(\theta - 120°) + 1}{3}\\
H &= \frac{2\cos(\theta + 120°) + 1}{3}
\end{align}

These coefficients arise from the tetrahedral geometry: the $120°$ phase shifts reflect the three-fold symmetry of the face opposite each vertex.

\subsection{RT-Pure Form Using Spread}

We rationalize Ace's formula using spread $s = \sin^2(\theta)$ and cross $c = \cos^2(\theta) = 1 - s$:

\begin{verbatim}
function rotationCoeffsFromSpread(s) {
  const cosTheta = Math.sqrt(1 - s);  // Deferred √ until needed
  const sinTheta = Math.sqrt(s);

  const cos120 = -0.5;              // Exact rational: -1/2
  const sin120 = Math.sqrt(0.75);   // √(3/4) - deferred

  const F = (2 * cosTheta + 1) / 3;
  const G = (2 * (cosTheta * cos120 + sinTheta * sin120) + 1) / 3;
  const H = (2 * (cosTheta * cos120 - sinTheta * sin120) + 1) / 3;

  return { F, G, H };
}
\end{verbatim}

\subsection{The Native 4×4 Rotation Matrix}

Rotation about the $W$-axis applies F, G, H in a \textbf{circulant} pattern to the other three coordinates:

\begin{equation}
R_W = \begin{pmatrix}
1 & 0 & 0 & 0\\
0 & F & H & G\\
0 & G & F & H\\
0 & H & G & F
\end{pmatrix}
\end{equation}

\begin{observation}[Circulant Structure]
The $3 \times 3$ submatrix acting on $(X, Y, Z)$ is \textbf{circulant}---each row is a cyclic permutation of the previous. This reflects the tetrahedral symmetry where all non-axis coordinates are treated equivalently around the rotation axis.
\end{observation}

For rotation about other axes, the matrix structure permutes accordingly:
\begin{itemize}
    \item $R_X$: row/column 1 fixed, circulant on $(W, Y, Z)$
    \item $R_Y$: row/column 2 fixed, circulant on $(W, X, Z)$
    \item $R_Z$: row/column 3 fixed, circulant on $(W, X, Y)$
\end{itemize}

\subsection{Why This Differs from Quaternions}

This is \textbf{not} ``quaternions in disguise.'' The key differences:

\begin{table}[h]
\centering
\begin{tabular}{lll}
\toprule
\textbf{Aspect} & \textbf{Quaternion Rotation} & \textbf{Native Quadray Rotation}\\
\midrule
Basis geometry & Orthogonal ($90°$) & Tetrahedral ($109.47°$)\\
Multiplication & Hamilton product & F,G,H matrix multiplication\\
Structure & $ij = k$ relations & Circulant matrix patterns\\
Axis representation & Unit vector in $\mathbb{R}^3$ & One of four basis directions\\
Constraint & $\|q\| = 1$ required & No norm constraint\\
\bottomrule
\end{tabular}
\caption{Structural differences between quaternion and native Quadray rotation}
\end{table}

The F,G,H coefficients emerge from the tetrahedral central angle ($109.47°$) and its three-fold face symmetry---algebraic structures that have no quaternion equivalent.

%==============================================================================
\section{On Normalization: XYZ Parity vs Native 4D}
%==============================================================================

\subsection{The Zero-Sum Constraint Revisited}

Standard Quadray implementations enforce $W + X + Y + Z = k$ (zero-sum normalization), which projects 4D coordinates onto 3D Cartesian space. This is a \textbf{compatibility constraint}, not an intrinsic property of tetrahedral geometry.

\begin{observation}[The Fourth Coordinate Carries Information]
Consider the Quadray positions $(1, 1, 1, 1)$ and $(1, 1, 1, 6)$. With zero-sum normalization, these project to equivalent 3D positions. But they describe fundamentally different tetrahedral configurations---the second represents a \textbf{deformed tetrahedron} stretched along the Z-axis.
\end{observation}

This is demonstrated in ARTexplorer's ``Quadray Tetrahedron (Deformed)'' polyhedron, which renders with \texttt{normalize: false} to preserve the stretched geometry.

\subsection{Implications for Rotation}

For rotation representation, we have two choices:

\begin{table}[h]
\centering
\begin{tabular}{lp{5cm}p{5cm}}
\toprule
\textbf{Approach} & \textbf{With Normalization} & \textbf{Without Normalization}\\
\midrule
DOF & 3 (XYZ-equivalent) & 4 (native Quadray)\\
Gimbal lock & Possible (if reduced to $SO(3)$) & Avoided (remains in $\mathbb{R}^4$)\\
Constraint & $W + X + Y + Z = k$ & None\\
Interpolation & Must respect constraint & Simple linear in $\mathbb{R}^4$\\
Physical meaning & ``Ordinary'' 3D space & Full tetrahedral 4D space\\
\bottomrule
\end{tabular}
\caption{Normalized vs.\ native Quadray for rotation}
\end{table}

\subsection{Connection to Janus Geometry}

As documented in the companion paper \emph{Geometric Janus Inversion} (Thomson, 2026), the zero-sum constraint is specifically for maintaining XYZ parity---it allows Quadray to serve as an isomorphic substitute for Cartesian coordinates.

When we operate in \textbf{native 4D Quadray space} (without the zero-sum constraint), the fourth coordinate carries independent geometric information. The space becomes $\mathbb{R}^4$ rather than a projection onto $\mathbb{R}^3$, and the topological obstructions that cause gimbal lock in 3-parameter systems no longer apply.

This is the deeper reason why Quadray rotors can avoid gimbal lock: they operate in the full 4D space that the tetrahedron naturally inhabits, not in a 3D projection constrained for Cartesian compatibility.

%==============================================================================
\section{Implementation Status}
%==============================================================================

\subsection{Current State (ARTexplorer, February 2026)}

The Spread-Quadray Rotor system is fully implemented:

\begin{table}[h]
\centering
\begin{tabular}{lcc}
\toprule
\textbf{Component} & \textbf{Status} & \textbf{RT-Pure?}\\
\midrule
\texttt{QuadrayRotor} class & Implemented & \textbf{Yes}\\
\texttt{fromSpreadAxis()} entry point & Implemented & \textbf{Yes}\\
Explicit Janus polarity ($\pm 1$) & Implemented & \textbf{Yes}\\
Half-angle via spread/cross & Implemented & \textbf{Yes}\\
\texttt{multiply()} composition & Implemented & \textbf{Yes} (polynomial)\\
\texttt{toMatrix3()} output & Implemented & \textbf{Yes} (polynomial)\\
F,G,H native rotation & \textbf{Verified} & \textbf{Yes}\\
\bottomrule
\end{tabular}
\caption{Implementation status of Quadray rotor components}
\end{table}

\subsection{F,G,H Verification (Phase 6.2 Complete)}

Tom Ace's F,G,H rotation coefficients have been \textbf{verified} to produce identical results to quaternion rotation for all four Quadray basis axes (W, X, Y, Z). The verification test (\texttt{fgh-verification-test.js}) confirms:

\begin{itemize}
    \item \textbf{W,Y axes}: Right-circulant matrix pattern $[F,H,G; G,F,H; H,G,F]$
    \item \textbf{X,Z axes}: Left-circulant matrix pattern $[F,G,H; H,F,G; G,H,F]$ (chirality-corrected)
    \item \textbf{Maximum error}: $< 10^{-15}$ (machine precision)
\end{itemize}

The Hamilton product remains as the composition method because it is itself \textbf{RT-pure}---pure polynomial multiplication with no transcendental functions. The ``scaffolding'' has proven to be the final structure.

\subsection{What's RT-Pure}

The implementation achieves RT-purity at multiple levels:

\begin{enumerate}
    \item \textbf{Entry point}: Spread $s = \sin^2\theta$ instead of angle $\theta$
    \item \textbf{Half-angle calculation}: Uses spread/cross algebra with deferred $\sqrt{}$
    \item \textbf{Composition}: Hamilton product is polynomial (no trig)
    \item \textbf{Output}: \texttt{toMatrix3()} produces quadratic polynomials in $(w,x,y,z)$
\end{enumerate}

Classical trig usage is limited to justified cases: degree/radian conversion for human-readable display, SLERP interpolation (standard algorithm), and frame update calculations.

%==============================================================================
\section{Interactive Demonstration}
%==============================================================================

ARTexplorer includes an interactive demo (\texttt{rt-rotor-demo.js}) that visualizes:

\begin{itemize}
    \item A spinning geodesic octahedron controlled by the Quadray rotor
    \item Draggable axis handle to adjust spin direction in real-time
    \item Gimbal lock zones (red warning disks at $\pm 90°$ Y)
    \item Color-coded proximity feedback (green $\to$ yellow $\to$ orange $\to$ red)
    \item Mode toggle between Euler XYZ and Quadray Rotor
\end{itemize}

\begin{figure}
    \centering
    \includegraphics[width=0.8\linewidth]{Euler-Catastrophe.png}
    \caption{XYZ Euler Gimbal Lock Failure is represented in the demo by warning regions and pulsing zones of failure. Interactive orbit handles allow the user to drag the primary axis of rotation, with a handle at the top of the axis also}
    \label{fig:placeholder}
\end{figure}
\textbf{
Key demonstration:} In Euler mode, approaching the gimbal lock zones causes visible rotation instability (simulated jitter). In Quadray mode, the same axis orientations produce smooth, continuous rotation---the zones are ``irrelevant'' because the 4D representation has no singularities.

\textbf{Note on gimbal lock zone visualization:} The current demo displays gimbal lock zones at the Y-axis poles ($\pm 90°$), which is correct for Euler XYZ rotation order specifically. In general, gimbal lock occurs when the \emph{middle axis} of any Euler rotation order reaches $\pm 90°$. A complete visualization would show six singularity regions---one at each face center of an inscribed cube:
\begin{itemize}
    \item $\pm$Y poles: XYZ and ZYX orders
    \item $\pm$X poles: YXZ and ZXY orders
    \item $\pm$Z poles: XZY and YZX orders
\end{itemize}
A future enhancement could map a cube in 3D space with gimbal lock zones on each face, allowing users to select different Euler orders and see the corresponding singularity locations.

\begin{figure}
    \centering
    \includegraphics[width=0.8\linewidth]{Quadray-Pass.png}
    \caption{In Quadray Rotor mode of the same demo, the user can see no gimbal lock failure in the same regions.}
    \label{fig:placeholder}
\end{figure}

The demo is available at: \url{https://arossti.github.io/ARTexplorer/}

%==============================================================================
\section{Quadray Basis Axes vs Cartesian Axes}
%==============================================================================

A critical distinction for understanding native Quadray rotation:

\subsection{The Tetrahedral Basis Directions}

The four Quadray basis vectors, expressed in Cartesian coordinates:
\begin{align}
\hat{W} &= \frac{1}{\sqrt{3}}(1, 1, 1)\\
\hat{X} &= \frac{1}{\sqrt{3}}(1, -1, -1)\\
\hat{Y} &= \frac{1}{\sqrt{3}}(-1, 1, -1)\\
\hat{Z} &= \frac{1}{\sqrt{3}}(-1, -1, 1)
\end{align}

These point to the vertices of a regular tetrahedron inscribed in the unit sphere. The angle between any two basis vectors is $\cos^{-1}(-1/3) \approx 109.47°$---the tetrahedral central angle.

\subsection{F,G,H Rotates About Quadray Axes, Not Cartesian}

Tom Ace's F,G,H formula rotates about these tetrahedral directions, NOT the Cartesian $\hat{x}, \hat{y}, \hat{z}$:

\begin{itemize}
    \item $R_W(\theta)$: Rotation about $(1,1,1)/\sqrt{3}$ (body diagonal of cube)
    \item $R_X(\theta)$: Rotation about $(1,-1,-1)/\sqrt{3}$
    \item $R_Y(\theta)$: Rotation about $(-1,1,-1)/\sqrt{3}$
    \item $R_Z(\theta)$: Rotation about $(-1,-1,1)/\sqrt{3}$
\end{itemize}

This is the native tetrahedral rotation system---fundamentally different from Cartesian XYZ.

\subsection{Equivalence Hypothesis}

We hypothesize that $R_W(\theta)$ applied to a point $P$ in Quadray coordinates produces the same 3D result as a quaternion rotation about $\hat{W} = (1,1,1)/\sqrt{3}$ by angle $\theta$:

\begin{equation}
\text{CartesianToQuadray}^{-1}(R_W(\theta) \cdot P) = q_{\hat{W},\theta} \cdot \text{CartesianToQuadray}^{-1}(P) \cdot q_{\hat{W},\theta}^{-1}
\end{equation}

If true, then F,G,H is an alternative notation for quaternion rotation about tetrahedral axes---not a new rotation system, but a coordinate-native expression of the same rotations.

%==============================================================================
\section{The Sign Ambiguity and Janus Resolution}
%==============================================================================

\subsection{Spread Doesn't Distinguish Direction}

Given spread $s = \sin^2(\theta)$:
\begin{align}
\cos(\theta) &= \pm\sqrt{1 - s}\\
\sin(\theta) &= \pm\sqrt{s}
\end{align}

For $\theta = 60°$ and $\theta = 120°$, the spread is identical ($s = 3/4$), yet the rotations differ!

\subsection{Janus Polarity Resolves the Ambiguity}

The explicit polarity flag $p \in \{+1, -1\}$ determines the sign of $\cos(\theta)$:

\begin{table}[h]
\centering
\begin{tabular}{cccl}
\toprule
\textbf{Polarity} $p$ & $\cos(\theta)$ & \textbf{Effective Range} & \textbf{Meaning}\\
\midrule
$+1$ & $+\sqrt{1-s}$ & $0° \leq \theta \leq 90°$ & ``Forward'' rotation\\
$-1$ & $-\sqrt{1-s}$ & $90° < \theta \leq 180°$ & ``Backward'' rotation\\
\bottomrule
\end{tabular}
\caption{Janus polarity resolves spread sign ambiguity}
\end{table}

\begin{observation}[Dual Purpose of Janus Polarity]
The polarity flag serves two functions:
\begin{enumerate}
    \item Explicit double-cover (replacing quaternion's implicit $q \equiv -q$)
    \item Sign disambiguation for spread-based calculations
\end{enumerate}
Both are necessary for RT-pure rotation to be well-defined.
\end{observation}

The RT-pure coefficient formula becomes:
\begin{verbatim}
function rotationCoeffsFromSpread(s, polarity = +1) {
  const cosTheta = polarity * Math.sqrt(1 - s);  // Sign from polarity
  const sinTheta = Math.sqrt(s);  // Always positive

  const cos120 = -0.5;
  const sin120 = Math.sqrt(0.75);

  const F = (2 * cosTheta + 1) / 3;
  const G = (2 * (cosTheta * cos120 + sinTheta * sin120) + 1) / 3;
  const H = (2 * (cosTheta * cos120 - sinTheta * sin120) + 1) / 3;

  return { F, G, H };
}
\end{verbatim}

%==============================================================================
\section{Composition: Matrix Product vs Hamilton Product}
%==============================================================================

\subsection{The Verification Question}

Does 4×4 matrix multiplication of F,G,H rotation matrices produce the same rotation as Hamilton product of the corresponding quaternions?

For \textbf{same-axis} composition:
\begin{equation}
R_W(\theta_1) \times R_W(\theta_2) \stackrel{?}{=} R_W(\theta_1 + \theta_2)
\end{equation}

This should hold because both methods simply add rotation angles about the same axis.

For \textbf{different-axis} composition:
\begin{equation}
R_W(\theta_1) \times R_X(\theta_2) \stackrel{?}{=} q_{\hat{W},\theta_1} \cdot q_{\hat{X},\theta_2} \text{ (converted to matrix)}
\end{equation}

This is the critical test. The 4×4 matrix product is well-defined, as is the Hamilton product. Do they agree?

\subsection{Three Possible Outcomes}

\begin{enumerate}
    \item \textbf{Identical}: F,G,H matrix multiplication $\equiv$ Hamilton product for all compositions.\\
    \emph{Implication}: Native Quadray rotation is quaternions in different coordinates---elegant but not novel.

    \item \textbf{Different}: The compositions produce different 3D rotations.\\
    \emph{Implication}: Native Quadray rotation is a genuinely different rotation system---novel but potentially problematic for matching expected behavior.

    \item \textbf{Partial}: Same for single-axis, different for composition.\\
    \emph{Implication}: The tetrahedral geometry introduces composition effects not present in quaternions---interesting hybrid.
\end{enumerate}

%==============================================================================
\section{Phase 6.0 Verification Results}
\label{sec:verification}
%==============================================================================

\subsection{Test Methodology}

A verification test suite (\texttt{fgh-verification-test.js}) was developed to compare F,G,H rotation in Quadray space against quaternion rotation in Cartesian space. The test protocol:

\begin{enumerate}
    \item Take a test point in Cartesian coordinates: $(1, 0.5, 0.3)$
    \item Convert to Quadray coordinates
    \item Apply F,G,H rotation about a basis axis
    \item Convert result back to Cartesian
    \item Compare against quaternion rotation about the same axis (expressed in Cartesian)
    \item Measure maximum coordinate error
\end{enumerate}

\subsection{Results Summary}

\begin{table}[h]
\centering
\begin{tabular}{lccc}
\toprule
\textbf{Test} & \textbf{Result} & \textbf{Max Error} & \textbf{Notes}\\
\midrule
W-axis 30° & \textcolor{green!50!black}{\textbf{PASS}} & $1.1 \times 10^{-16}$ & Machine precision\\
W-axis 45° & \textcolor{green!50!black}{\textbf{PASS}} & $2.2 \times 10^{-16}$ & Machine precision\\
W-axis 60° & \textcolor{green!50!black}{\textbf{PASS}} & $2.2 \times 10^{-16}$ & Machine precision\\
W-axis 90° & \textcolor{green!50!black}{\textbf{PASS}} & $1.7 \times 10^{-16}$ & Machine precision\\
W-axis 120° & \textcolor{green!50!black}{\textbf{PASS}} & $5.6 \times 10^{-17}$ & Machine precision\\
W-axis composition & \textcolor{green!50!black}{\textbf{PASS}} & $5.6 \times 10^{-17}$ & $R_W(45°) \times R_W(45°) = R_W(90°)$\\
Y-axis (all angles) & \textcolor{green!50!black}{\textbf{PASS}} & $\sim 10^{-16}$ & Right-circulant pattern\\
X-axis (all angles) & \textcolor{green!50!black}{\textbf{PASS}} & $\sim 10^{-16}$ & Left-circulant (Phase 6.2)\\
Z-axis (all angles) & \textcolor{green!50!black}{\textbf{PASS}} & $\sim 10^{-16}$ & Left-circulant (Phase 6.2)\\
Cross-axis composition & \textcolor{green!50!black}{\textbf{PASS}} & $\sim 10^{-16}$ & $R_W \times R_X$ matches quaternions\\
\bottomrule
\end{tabular}
\caption{Phase 6.0/6.2 verification test results (February 2026)}
\end{table}

\subsection{Key Findings}

\begin{tcolorbox}[colback=green!5, colframe=green!40!black, title=\textbf{Phase 6.2 Complete: All Four Quadray Axes Verified}]
\textbf{Outcome 1 confirmed for ALL axes}: F,G,H rotation produces \emph{identical} results to quaternion rotation about all four tetrahedral basis axes, to machine precision ($10^{-16}$).

\textbf{Key discovery}: Tetrahedral chirality requires two circulant patterns:
\begin{itemize}
    \item \textbf{Right-circulant} $[F, H, G; G, F, H; H, G, F]$: W and Y axes
    \item \textbf{Left-circulant} $[F, G, H; H, F, G; G, H, F]$: X and Z axes
\end{itemize}
The difference is swapping G and H positions to account for opposite handedness.
\end{tcolorbox}

\textbf{Detailed findings:}

\begin{enumerate}
    \item \textbf{W-axis F,G,H verified}: Tom Ace's original formula produces rotations identical to $q_{\hat{W},\theta}$ where $\hat{W} = (1,1,1)/\sqrt{3}$.

    \item \textbf{Y-axis also works}: The same right-circulant matrix pattern applies to Y-axis rotation.

    \item \textbf{X,Z-axis matrices SOLVED (Phase 6.2)}: The tetrahedral vertex arrangement has chirality---W and Y share one handedness, while X and Z have the opposite. Swapping G and H positions (left-circulant) produces correct rotations for X and Z axes.

    \item \textbf{Same-axis composition verified}: $R_W(45°) \times R_W(45°) = R_W(90°)$ exactly.

    \item \textbf{Cross-axis composition verified}: $R_W(45°) \times R_X(45°)$ matches quaternion composition to machine precision, confirming the rotation algebra is equivalent.
\end{enumerate}

\subsection{Mathematical Interpretation}

The verification establishes that for W-axis rotation:
\begin{equation}
\text{Quadray}^{-1}\bigl(R_W(\theta) \cdot \text{Quadray}(P)\bigr) = q_{\hat{W},\theta} \cdot P \cdot q_{\hat{W},\theta}^{-1}
\end{equation}

where $\text{Quadray}(\cdot)$ denotes the Cartesian-to-Quadray coordinate transformation. The F,G,H coefficients encode the same rotation as the quaternion, but expressed natively in tetrahedral coordinates.

\subsection{Implementation Roadmap (Updated)}

Based on verification results:

\begin{itemize}
    \item \textbf{Phase 6.1} (\textcolor{green!50!black}{\textbf{COMPLETE}}): Implement W/Y-axis F,G,H rotation in \texttt{rt-math.js}
    \item \textbf{Phase 6.2} (\textcolor{green!50!black}{\textbf{COMPLETE}}): Derive X/Z axis matrices via left-circulant pattern
    \item \textbf{Phase 6.3} (\textcolor{green!50!black}{\textbf{COMPLETE}}): Integrate into demo with interactive F,G,H rotation buttons for all four axes
    \item \textbf{Phase 6.4}: Performance comparison vs quaternion path
    \item \textbf{Phase 6.5} (\textcolor{green!50!black}{\textbf{COMPLETE}}): Interactive orbit handles for dragging spin axis with mode-based visibility
\end{itemize}

%==============================================================================
\section{Remaining Open Questions}
%==============================================================================

\begin{enumerate}
    \item \textbf{[ANSWERED]} \emph{Equivalence verification:} Does $R_W(\theta)$ match $q_{\hat{W},\theta}$?\\
    \textbf{Yes}, to machine precision. F,G,H is a coordinate-native expression of quaternion rotation.

    \item \textbf{[ANSWERED]} \emph{X,Z-axis matrix structure:} What is the correct F,G,H placement for X and Z axes?\\
    \textbf{Left-circulant} (G,H swapped): tetrahedral chirality requires opposite handedness for X,Z vs W,Y.

    \item \textbf{[ANSWERED]} \emph{Arbitrary axis rotation:} Can we decompose arbitrary axis rotation into W,X,Y,Z compositions? Does this introduce ``tetrahedral gimbal lock''?\\
    \textbf{Yes}, arbitrary axis rotation works via basis composition. The interactive demo (Phase 6.5) allows dragging the spin axis to any orientation using orbit handles. No ``tetrahedral gimbal lock'' observed---the 4D representation remains smooth everywhere, unlike Euler angles which exhibit jitter near $\pm 90°$ Y-axis.

    \item \textbf{[PARTIALLY ANSWERED]} \emph{Normalization trade-offs:} Operating without zero-sum normalization preserves 4 DOF. Are there numerical stability concerns?\\
    \textbf{No stability issues observed} in practice. The demo runs continuously with Hamilton product composition and periodic normalization. Long-running sessions show no drift or instability. Further stress testing recommended for edge cases.

    \item \textbf{Closed-form composition rule:} Can we derive explicit formulas for $F_{12}, G_{12}, H_{12}$ from $R(\theta_1) \times R(\theta_2)$?\\
    \emph{Open question.} Current implementation uses Hamilton product for composition, which is verified correct. Deriving native F,G,H composition formulas remains a theoretical goal.

    \item \textbf{[ANSWERED]} \emph{GPU efficiency:} Can F,G,H 4×4 matrices be sent directly to shaders, or is quaternion conversion at the boundary necessary?\\
    \textbf{Quaternion-to-matrix conversion at boundary.} The current implementation converts QuadrayRotor to THREE.js Quaternion for rendering. The \texttt{toMatrix3()} method produces RT-pure polynomial output (no transcendentals). Direct 4×4 matrix shaders remain a future optimization.

    \item \textbf{Weierstrass-based rotor construction:} Can rotors be constructed directly from Weierstrass parameter $t = \tan(\theta/2)$ instead of spread?\\
    \emph{Research direction.} The current \texttt{fromSpreadAxis()} uses half-angle identities with $\sqrt{}$ operations. For rational $t$, Weierstrass gives algebraic $\cos\theta$ and $\sin\theta$. Extending this to $\cos(\theta/2)$, $\sin(\theta/2)$ for rotor construction could eliminate all $\sqrt{}$ for a subset of useful rotations. See Section~\ref{sec:weierstrass}.
\end{enumerate}

%==============================================================================
\section{Conclusion}
%==============================================================================

We have presented Spread-Quadray Rotors as a novel alternative to quaternions for gimbal-lock-free rotation representation. The key innovations are:

\begin{enumerate}
    \item \textbf{Tetrahedral basis geometry}---$109.47°$ angles rather than orthogonal $90°$
    \item \textbf{Spread/cross measures}---rational values for many useful rotations
    \item \textbf{No norm constraint}---freedom in $\mathbb{R}^4$ rather than confinement to $S^3$
    \item \textbf{Explicit Janus polarity}---discrete double-cover rather than implicit $q \equiv -q$
    \item \textbf{Algebraic exactness}---transcendentals deferred until GPU boundary
\end{enumerate}

The Hairy Ball Theorem guarantees that any 3-parameter rotation system will have singularities. Quaternions escape via the hypersphere $S^3$; Quadray rotors escape via $\mathbb{R}^4 \times \mathbb{Z}_2$. Both are valid lifts from the twisted topology of $SO(3)$, but they produce fundamentally different algebraic structures.

Whether Quadray rotors offer practical advantages over quaternions in specific applications remains to be determined through implementation and testing. The theoretical framework, however, is complete and internally consistent.

%==============================================================================
\section*{Acknowledgments}
%==============================================================================

This work draws on:
\begin{itemize}
    \item N.J.\ Wildberger's Rational Trigonometry
    \item Kirby Urner's Quadray coordinate system
    \item Tom Ace's Quadray rotation formulas
    \item R.\ Buckminster Fuller's synergetic geometry
    \item Julian Barbour's Janus Point concept
\end{itemize}

Developed in collaboration with AI assistance (Claude/Anthropic) for mathematical formalization and documentation.

\theendnotes

%==============================================================================
\section*{References}
%==============================================================================

\begin{enumerate}
    \item Brouwer, L.E.J. (1912). ``Über Abbildung von Mannigfaltigkeiten.'' \emph{Mathematische Annalen}.
    \item Wildberger, N.J. (2005). \emph{Divine Proportions: Rational Trigonometry to Universal Geometry}. Wild Egg Books.
    \item Shoemake, K. (1985). ``Animating Rotation with Quaternion Curves.'' \emph{SIGGRAPH}.
    \item Hanson, A.J. (2006). \emph{Visualizing Quaternions}. Morgan Kaufmann.
    \item Fuller, R.B. (1975). \emph{Synergetics: Explorations in the Geometry of Thinking}. Macmillan.
    \item Barbour, J. (2020). \emph{The Janus Point: A New Theory of Time}. Basic Books.
    \item Urner, K. ``Quadray Coordinates: A Logical Alternative.'' \url{http://www.grunch.net/synergetics/quadintro.html}
    \item Ace, T. ``Quadray Coordinates.'' \url{http://minortriad.com/quadray.html}
    \item Thomson, A. (2026). \emph{ARTexplorer: Interactive Tetrahedral Geometry Visualization}. \url{https://arossti.github.io/ARTexplorer/}
    \item Thomson, A. (2026). ``Geometric Janus Inversion: Extending the Janus Point from Temporal to Spatial Geometry via Tetrahedral (Quadray) Coordinates.'' Open Building / ARTexplorer Project. (Companion paper on 4D$^\pm$ Quadray space and the zero-sum normalization constraint.)
    \item DOI: 10.13140/RG.2.2.23476.51846 | CC BY-NC-ND 4.0 | \copyright\ Andy Thomson 2026
\end{enumerate}

\vfill
\begin{center}
\textit{``You cannot comb a hairy ball flat without creating a cowlick.''\\[0.5em]
--- L.E.J.\ Brouwer (1912), as applied to rotation space}
\end{center}

\end{document}
