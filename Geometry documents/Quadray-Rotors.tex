\documentclass[11pt,a4paper]{article}
\usepackage[utf8]{inputenc}
\usepackage[T1]{fontenc}
\usepackage{amsmath,amssymb,amsthm}
\usepackage{geometry}
\usepackage{hyperref}
\usepackage{graphicx}
\usepackage{booktabs}
\usepackage{enumitem}
\usepackage{xcolor}
\usepackage{parskip}  % Adds vertical space between paragraphs
\usepackage{tcolorbox}  % For framed boxes
\usepackage{endnotes}   % For endnotes instead of footnotes

\geometry{margin=1in}

\hypersetup{
    colorlinks=true,
    linkcolor=blue,
    urlcolor=blue,
    citecolor=blue
}

\newtheorem{conjecture}{Conjecture}
\newtheorem{definition}{Definition}
\newtheorem{observation}{Observation}
\newtheorem{theorem}{Theorem}
\newtheorem{proposition}{Proposition}

\title{Spread-Quadray Rotors -- v1.0 (Feb 2026)\\
\large A Tetrahedral Alternative to Quaternions\\
for Gimbal-Lock-Free Rotation Representation}

\author{Andrew Thomson\\
\small Open Building / ARTexplorer Project\\
\small \href{mailto:andy@openbuilding.ca}{andy@openbuilding.ca}}

\date{February 2026}

\begin{document}

\maketitle

\begin{abstract}
This document proposes a novel approach to 3D rotation representation: \textbf{Spread-Quadray Rotors}. By combining full 4D Quadray coordinates (without zero-sum constraint), Rational Trigonometry (spread/cross instead of sin/cos), Weierstrass parametrization, and explicit Janus polarity, we arrive at a rotation system that is gimbal-lock free, algebraically exact for many useful rotations, and geometrically native to tetrahedral structures.

The \textbf{Hairy Ball Theorem} (Brouwer, 1912) establishes that any continuous tangent vector field on a sphere must have at least one singularity---you cannot ``comb a hairy ball flat without creating a cowlick.'' This topological obstruction is why 3-parameter rotation representations (Euler angles, zero-sum Quadray) inevitably suffer gimbal lock. Quaternions escape this by lifting to the 4-dimensional hypersphere $S^3$; we propose an alternative lift to $\mathbb{R}^4 \times \mathbb{Z}_2$ via tetrahedral coordinates, yielding a fundamentally different algebraic structure that preserves exact rational values throughout rotation calculations.

This is not merely ``quaternions in different clothing'' but a genuinely distinct representation arising from tetrahedral geometry and Wildberger's Rational Trigonometry.
\end{abstract}

\tableofcontents
\newpage

%==============================================================================
\section{Introduction}
%==============================================================================

\subsection{The Problem: Gimbal Lock}

Gimbal lock occurs when two rotation axes align, causing a loss of one degree of freedom. In Euler angle representation (pitch, yaw, roll), this happens at $\pm 90°$ pitch---the system loses the ability to distinguish yaw from roll.

\textbf{Physical analogy:} A camera gimbal with three nested rings. When the middle ring rotates $90°$, the inner and outer rings become parallel---you can no longer independently control two of the three rotations.

\textbf{Why it matters:}
\begin{itemize}
    \item \textbf{Animation:} Interpolating between orientations near gimbal lock produces erratic motion
    \item \textbf{Robotics:} Control systems can become unstable
    \item \textbf{Aerospace:} Apollo 11's guidance computer nearly encountered gimbal lock
    \item \textbf{3D Graphics:} SLERP requires quaternions to avoid interpolation artifacts
\end{itemize}

\subsection{The Standard Solution: Quaternions}

Quaternions avoid gimbal lock by using 4 parameters constrained to a 3-sphere ($S^3$):
\begin{equation}
q = q_0 + q_1 i + q_2 j + q_3 k \quad \text{where} \quad q_0^2 + q_1^2 + q_2^2 + q_3^2 = 1
\end{equation}

This works because $S^3$ is \emph{simply connected}---any two orientations can be connected by a continuous path without passing through a singularity.

\textbf{But quaternions have limitations:}
\begin{itemize}
    \item Require transcendental functions (sin, cos) for most angle values
    \item The unit norm constraint requires renormalization after composition
    \item The double-cover ($q$ and $-q$ represent the same rotation) is implicit, not explicit
\end{itemize}

\subsection{Our Proposal: Spread-Quadray Rotors}

We propose an alternative 4-parameter rotation representation based on:
\begin{enumerate}
    \item \textbf{Full 4D Quadray coordinates}---without zero-sum constraint
    \item \textbf{Spread/Cross measures}---from Wildberger's Rational Trigonometry
    \item \textbf{Weierstrass parametrization}---for exact rational circle points
    \item \textbf{Janus polarity}---explicit discrete state for double-cover
\end{enumerate}

%==============================================================================
\section{Topological Foundations}
%==============================================================================

\subsection{The Hairy Ball Theorem}

\begin{theorem}[Hairy Ball Theorem --- Brouwer, 1912]
There is no nonvanishing continuous tangent vector field on even-dimensional $n$-spheres. In particular, on $S^2$, any continuous tangent vector field must have at least one point where the vector vanishes.
\end{theorem}

\textbf{Colloquial version:} ``You cannot comb a hairy ball flat without creating at least one cowlick.''

\textbf{Implications:}
\begin{itemize}
    \item There is always at least one point on Earth with zero wind (a cyclone eye or calm spot)
    \item You cannot texture-map a sphere without at least one singularity
    \item \textbf{Gimbal lock is unavoidable with 3-parameter rotation systems}
\end{itemize}

\subsection{Why 3 Parameters Are Not Enough}

The space of 3D rotations is $SO(3)$---a 3-dimensional manifold that is \emph{not simply connected}. Topologically, $SO(3)$ is equivalent to $\mathbb{RP}^3$ (real projective 3-space), which has a ``twist'' that prevents global singularity-free parameterization with only 3 numbers.

\begin{observation}[Topological Obstruction]
You cannot parameterize $SO(3)$ with 3 continuous parameters without creating at least one singularity. This is a direct consequence of the Hairy Ball Theorem applied to rotation space.
\end{observation}

\subsection{The Lift to 4D}

The solution is to \textbf{lift} from $SO(3)$ to a higher-dimensional space:

\begin{table}[h]
\centering
\begin{tabular}{cccc}
\toprule
\textbf{Space} & \textbf{Dimension} & \textbf{Topology} & \textbf{Singularities}\\
\midrule
$SO(3)$ & 3 & $\mathbb{RP}^3$ (twisted) & Unavoidable\\
$S^3$ (Quaternions) & 3-sphere in $\mathbb{R}^4$ & Simply connected & None\\
$\mathbb{R}^4$ (Full Quadray) & 4D Euclidean & Simply connected & None\\
\bottomrule
\end{tabular}
\caption{Topological comparison of rotation representation spaces}
\end{table}

Both $S^3$ (quaternions) and $\mathbb{R}^4$ (full Quadray) provide this lift, but with different geometric structures.

%==============================================================================
\section{Quadray vs Quaternion Topology}
%==============================================================================

\subsection{Comparison of Representations}

\begin{table}[h]
\centering
\small
\begin{tabular}{lccccc}
\toprule
\textbf{Representation} & \textbf{Scalars} & \textbf{Constraint} & \textbf{Manifold} & \textbf{Lift} & \textbf{Gimbal Lock}\\
\midrule
Euler angles & 3 & None & $SO(3)$ chart & None & \textbf{Yes}\\
Quadray (zero-sum) & $4 \to 3$ & $w+x+y+z = k$ & $SO(3)$ chart & None & \textbf{Yes}\\
\textbf{Quadray (full 4D)} & 4 & None & $\mathbb{R}^4$ & Implicit $\mathbb{R}^4$ & \textbf{No}$^*$\\
Quaternions & 4 & $\|q\| = 1$ & $S^3$ & Explicit Spin(3) & \textbf{No}\\
\bottomrule
\end{tabular}
\caption{Comparison of rotation representations. $^*$Gimbal lock avoided provided the 4 scalars parameterize orientation directly and are not reduced to a 3-parameter $SO(3)$ chart.}
\end{table}

\subsection{The Zero-Sum Trap}

\begin{observation}[Critical Distinction]
When we enforce $W + X + Y + Z = \text{constant}$, we \textbf{project} 4D Quadray back to 3D. This projection reintroduces the $SO(3)$ topology and its inherent singularities.
\end{observation}

\begin{equation}
\text{Full Quadray } (\mathbb{R}^4) \xrightarrow{\text{zero-sum constraint}} \text{Projected Quadray } (\mathbb{R}^3) \cong SO(3)
\end{equation}

\textbf{The zero-sum constraint is a projection, not a necessity.} To avoid gimbal lock, we must work in the full 4D space.

\subsection{Why Full Quadray Differs from Quaternions}

\begin{table}[h]
\centering
\begin{tabular}{lll}
\toprule
\textbf{Aspect} & \textbf{Quaternions} & \textbf{Full Quadray}\\
\midrule
Basis geometry & Orthogonal ($90°$ between $i, j, k$) & Tetrahedral ($109.47°$ between $W, X, Y, Z$)\\
Constraint & Unit norm (forces onto $S^3$) & None (free in $\mathbb{R}^4$)\\
Double-cover & Implicit ($q \equiv -q$) & Explicit (Janus polarity bit)\\
Composition & Hamilton product & Tetrahedral rotation matrices\\
Interpolation & SLERP on $S^3$ & Linear in $\mathbb{R}^4$ (simpler!)\\
\bottomrule
\end{tabular}
\caption{Structural differences between quaternions and full Quadray}
\end{table}

%==============================================================================
\section{Spread-Quadray Rotors: Definition}
%==============================================================================

\subsection{The Core Concept}

Instead of using angle $\theta$ with sin/cos, we use:
\begin{itemize}
    \item \textbf{Spread} $s = \sin^2(\theta)$---a rational value for many useful angles
    \item \textbf{Cross} $c = \cos^2(\theta) = 1 - s$---the complementary measure
    \item \textbf{Weierstrass parameter} $t$---generates exact rational sin/cos values
\end{itemize}

\begin{definition}[Spread-Quadray Rotor]
A \textbf{Spread-Quadray Rotor} $R$ is defined as:
\begin{equation}
R = (W, X, Y, Z, \pm) \in \mathbb{R}^4 \times \mathbb{Z}_2
\end{equation}
where:
\begin{itemize}
    \item $(W, X, Y, Z)$ are four independent scalars (no zero-sum constraint)
    \item $\pm$ is the \textbf{Janus polarity} (discrete: positive or negative dimensional space)
\end{itemize}
\end{definition}

\subsection{Rotor Parameters from Spread}

Given spread $s$ and Weierstrass parameter $t$:
\begin{align}
t &= \sqrt{\frac{s}{1-s}} \quad \text{(parameter from spread)}\\
\cos(\theta) &= \frac{1 - t^2}{1 + t^2} \quad \text{(algebraic, no transcendentals!)}\\
\sin(\theta) &= \frac{2t}{1 + t^2} \quad \text{(algebraic, no transcendentals!)}
\end{align}

\subsection{Why ``Rotor'' Not ``Quaternion''}

We deliberately avoid calling these ``Quadray quaternions'' because:
\begin{enumerate}
    \item \textbf{Different algebra:} Quaternions use Hamilton multiplication ($ij = k$, etc.). Quadray rotors use tetrahedral rotation matrices.
    \item \textbf{Different constraint:} Quaternions require $\|q\| = 1$. Quadray rotors have no norm constraint.
    \item \textbf{Different topology:} Quaternions live on $S^3$. Quadray rotors live in $\mathbb{R}^4 \times \mathbb{Z}_2$.
    \item \textbf{Different exactness:} Quaternions require transcendentals for most angles. Quadray rotors can be exact rational for many useful rotations.
\end{enumerate}

%==============================================================================
\section{RT-Pure Rotation Mathematics}
%==============================================================================

\subsection{Spread and Cross}

From Wildberger's Rational Trigonometry:
\begin{align}
\text{Spread:} \quad s &= \sin^2(\theta) \quad \text{(measures ``perpendicularity'': 0 = parallel, 1 = perpendicular)}\\
\text{Cross:} \quad c &= \cos^2(\theta) \quad \text{(complementary measure)}\\
\text{Identity:} \quad s + c &= 1
\end{align}

\begin{observation}[Rational Spreads]
Spread is often a \textbf{rational number} even when $\sin(\theta)$ is irrational.
\end{observation}

\begin{table}[h]
\centering
\begin{tabular}{ccccc}
\toprule
\textbf{Angle} $\theta$ & $\sin(\theta)$ & $\cos(\theta)$ & \textbf{Spread} $s$ & \textbf{Cross} $c$\\
\midrule
$0°$ & 0 & 1 & 0 & 1\\
$30°$ & $1/2$ & $\sqrt{3}/2$ & \textbf{1/4} & \textbf{3/4}\\
$45°$ & $\sqrt{2}/2$ & $\sqrt{2}/2$ & \textbf{1/2} & \textbf{1/2}\\
$60°$ & $\sqrt{3}/2$ & $1/2$ & \textbf{3/4} & \textbf{1/4}\\
$90°$ & 1 & 0 & \textbf{1} & \textbf{0}\\
\bottomrule
\end{tabular}
\caption{Spread and cross values for common angles---note that all spreads are exact rationals}
\end{table}

\subsection{Weierstrass Parametrization}

The Weierstrass substitution provides \textbf{algebraic} circle points:
\begin{align}
t &= \tan(\theta/2) \quad \text{(the parameter)}\\
\cos(\theta) &= \frac{1 - t^2}{1 + t^2}\\
\sin(\theta) &= \frac{2t}{1 + t^2}
\end{align}

\begin{theorem}[RT-Pure Benefit]
For any \textbf{rational} $t$, both $\cos(\theta)$ and $\sin(\theta)$ are \textbf{exact rational values}.
\end{theorem}

This is implemented in ARTexplorer's \texttt{rt-math.js}:
\begin{verbatim}
RT.circleParam = t => {
  const tSquared = t * t;
  const denominator = 1 + tSquared;
  return {
    x: (1 - tSquared) / denominator,  // cos(theta) -- algebraic!
    y: (2 * t) / denominator,         // sin(theta) -- algebraic!
  };
};
\end{verbatim}

\subsection{From Spread to Weierstrass Parameter}

Given spread $s$, find parameter $t$:
\begin{align}
s &= \sin^2(\theta) = \left[\frac{2t}{1+t^2}\right]^2 = \frac{4t^2}{(1+t^2)^2}\\
\text{Solving:} \quad s(1 + t^2)^2 &= 4t^2\\
s \cdot t^4 + (2s - 4)t^2 + s &= 0
\end{align}

Using the quadratic formula with $u = t^2$:
\begin{align}
u &= \frac{4 - 2s \pm \sqrt{16 - 16s}}{2s} = \frac{2 - s \pm 2\sqrt{1-s}}{s}\\
t &= \sqrt{u} = \sqrt{\frac{2 - s + 2\sqrt{1-s}}{s}} \quad \text{(taking positive root)}
\end{align}

For exact rational spreads, this often simplifies beautifully.

%==============================================================================
\section{The Tetrahedral Rotation Matrix}
%==============================================================================

\subsection{Tom Ace's Quadray Rotation Formula}

Rotation about a Quadray axis uses coefficients $F$, $G$, $H$:
\begin{align}
F &= \frac{2\cos(\theta) + 1}{3}\\
G &= \frac{2\cos(\theta - 120°) + 1}{3}\\
H &= \frac{2\cos(\theta + 120°) + 1}{3}
\end{align}

\subsection{RT-Pure Form (Using Spread)}

\begin{verbatim}
function rotationCoeffsFromSpread(s) {
  const c = 1 - s;              // cross = cos^2(theta)
  const cosTheta = Math.sqrt(c);  // Deferred sqrt until needed
  const sinTheta = Math.sqrt(s);

  const cos120 = -0.5;           // -1/2 exactly
  const sin120 = Math.sqrt(0.75); // sqrt(3/4)

  const F = (2 * cosTheta + 1) / 3;
  const G = (2 * (cosTheta * cos120 + sinTheta * sin120) + 1) / 3;
  const H = (2 * (cosTheta * cos120 - sinTheta * sin120) + 1) / 3;

  return { F, G, H };
}
\end{verbatim}

\subsection{The 4×4 Rotation Matrix}

Rotation about the $W$-axis by spread $s$:
\begin{equation}
R = \begin{pmatrix}
1 & 0 & 0 & 0\\
0 & F & H & G\\
0 & G & F & H\\
0 & H & G & F
\end{pmatrix}
\end{equation}

\begin{observation}[Circulant Structure]
Note the \textbf{circulant structure} of the $3 \times 3$ submatrix---this reflects the tetrahedral symmetry where all non-axis coordinates are treated equivalently.
\end{observation}

%==============================================================================
\section{Exact Rational Rotations}
%==============================================================================

\subsection{The Gold Standard}

For certain angles, spread and cross are \textbf{exact rationals}, enabling algebraically exact rotation:

\begin{table}[h]
\centering
\begin{tabular}{ccccccl}
\toprule
\textbf{Rotation} & \textbf{Spread} $s$ & \textbf{Cross} $c$ & $F$ & $G$ & $H$ & \textbf{Notes}\\
\midrule
$0°$ & 0 & 1 & 1 & 0 & 0 & Identity\\
$30°$ & $1/4$ & $3/4$ & $(\sqrt{3}+1)/3$ & ... & ... & ---\\
$45°$ & $1/2$ & $1/2$ & $(\sqrt{2}+1)/3$ & ... & ... & Exact spread!\\
$60°$ & $3/4$ & $1/4$ & $2/3$ & $(\sqrt{3}+1)/6$ & $(1-\sqrt{3})/6$ & Tetrahedral\\
$90°$ & 1 & 0 & $1/3$ & $(\sqrt{3}+1)/3$ & $(1-\sqrt{3})/3$ & Quarter turn\\
$120°$ & $3/4$ & $1/4$ & 0 & $1/3$ & $1/3$ & \textbf{All rational!}\\
$180°$ & 0 & 1 & $-1/3$ & $2/3$ & $2/3$ & \textbf{All rational!}\\
\bottomrule
\end{tabular}
\caption{Rotation coefficients for common angles}
\end{table}

\subsection{Special Case: 120° Rotation (Tetrahedral Symmetry)}

At $120°$, the rotation coefficients become beautifully simple:
\begin{equation}
F = 0, \quad G = \frac{1}{3}, \quad H = \frac{1}{3}
\end{equation}

The rotation matrix becomes:
\begin{equation}
R_{120°} = \begin{pmatrix}
1 & 0 & 0 & 0\\
0 & 0 & 1/3 & 1/3\\
0 & 1/3 & 0 & 1/3\\
0 & 1/3 & 1/3 & 0
\end{pmatrix}
\end{equation}

This is a \textbf{cyclic permutation} of the non-axis coordinates---pure tetrahedral symmetry!

\subsection{Special Case: 180° Rotation (Janus Inversion)}

At $180°$, we get another exact rational matrix:
\begin{equation}
F = -\frac{1}{3}, \quad G = \frac{2}{3}, \quad H = \frac{2}{3}
\end{equation}

This represents the \textbf{Janus Point passage}---inversion through the origin from $4D^+$ to $4D^-$ space.

%==============================================================================
\section{The Janus Polarity Extension}
%==============================================================================

\subsection{Beyond Continuous Parameters}

Quaternions have an implicit double-cover: both $q$ and $-q$ represent the same rotation. This is mathematically elegant but can cause confusion (interpolation taking the ``long way around'').

Quadray rotors make this \textbf{explicit} with a discrete polarity bit:
\begin{equation}
\text{Full specification:} \quad (W, X, Y, Z, \pm)
\end{equation}

\begin{table}[h]
\centering
\begin{tabular}{ccc}
\toprule
\textbf{Polarity} & \textbf{Dimensional Space} & \textbf{Interpretation}\\
\midrule
$+$ & $4D^+$ & ``Positive'' dimensional space (outward from origin)\\
$-$ & $4D^-$ & ``Negative'' dimensional space (through origin, inverted)\\
\bottomrule
\end{tabular}
\caption{Janus polarity states}
\end{table}

The Janus Point (origin) is the \textbf{transition} between these spaces---passing through it flips the polarity.

\subsection{Advantages of Explicit Polarity}

\begin{enumerate}
    \item \textbf{Unambiguous interpolation:} Always know which ``sheet'' you're on
    \item \textbf{Cleaner animation:} No sudden flips when $q$ crosses $-q$ boundary
    \item \textbf{Physical meaning:} Connects to Fuller's ``inside-outing'' and cosmological Janus Point theories
    \item \textbf{Explicit topology:} The $\mathbb{Z}_2$ factor makes the double-cover visible
\end{enumerate}

%==============================================================================
\section{Implementation Roadmap}
%==============================================================================

\subsection{Current State (rt-math.js)}

ARTexplorer already implements key building blocks:
\begin{itemize}
    \item \texttt{RT.spread(v1, v2)}---Spread between vectors
    \item \texttt{RT.circleParam(t)}---Weierstrass parametrization
    \item \texttt{RT.spreadToParam(s)}---Convert spread to Weierstrass $t$
    \item \texttt{Quadray.toCartesian()}---Conversion (but uses zero-sum)
    \item \texttt{Quadray.fromCartesian()}---Reverse conversion
\end{itemize}

\subsection{Phase 1: Full 4D Quadray Storage}

\begin{verbatim}
class QuadrayRotor {
  constructor(w, x, y, z, polarity = '+') {
    this.w = w;
    this.x = x;
    this.y = y;
    this.z = z;
    this.polarity = polarity;  // '+' or '-'
  }
}
\end{verbatim}

\subsection{Phase 2: RT-Pure Rotation}

\begin{verbatim}
QuadrayRotor.prototype.rotateAboutW = function(spread) {
  const { F, G, H } = RT.rotationCoeffsFromSpread(spread);
  return new QuadrayRotor(
    this.w,
    F * this.x + H * this.y + G * this.z,
    G * this.x + F * this.y + H * this.z,
    H * this.x + G * this.y + F * this.z,
    this.polarity
  );
};
\end{verbatim}

\subsection{Phase 3: Composition and Interpolation}

\begin{verbatim}
// Linear interpolation in R^4 (simpler than SLERP!)
QuadrayRotor.lerp = function(r1, r2, t) {
  return new QuadrayRotor(
    r1.w + t * (r2.w - r1.w),
    r1.x + t * (r2.x - r1.x),
    r1.y + t * (r2.y - r1.y),
    r1.z + t * (r2.z - r1.z),
    t < 0.5 ? r1.polarity : r2.polarity
  );
};
\end{verbatim}

%==============================================================================
\section{Summary Comparison}
%==============================================================================

\begin{table}[h]
\centering
\begin{tabular}{lccc}
\toprule
\textbf{Feature} & \textbf{Euler Angles} & \textbf{Quaternions} & \textbf{Quadray Rotors}\\
\midrule
Parameters & 3 & 4 (constrained) & 4 + 1 discrete\\
Manifold & $SO(3)$ & $S^3$ & $\mathbb{R}^4 \times \mathbb{Z}_2$\\
Gimbal lock & Yes & No & No\\
Exact rationals & Rarely & Rarely & Often\\
Transcendentals & Always & Usually & Deferred\\
Interpolation & Problematic & SLERP & Linear\\
Double-cover & N/A & Implicit & Explicit\\
Native geometry & Orthogonal & Orthogonal & \textbf{Tetrahedral}\\
\bottomrule
\end{tabular}
\caption{Summary comparison of rotation representations}
\end{table}

%==============================================================================
\section{Open Questions}
%==============================================================================

\begin{enumerate}
    \item \textbf{Composition algebra:} Is there a ``Hamilton-like'' product for Quadray rotors, or must we use matrix multiplication?

    \item \textbf{Normalization:} Should we impose any constraint (analogous to unit quaternions) for numerical stability?

    \item \textbf{Optimal interpolation:} Is linear interpolation in $\mathbb{R}^4$ truly sufficient, or do we need something more sophisticated?

    \item \textbf{Physical interpretation:} What does a ``rotation by spread $3/4$'' mean geometrically in tetrahedral terms?

    \item \textbf{Performance:} Can RT-pure rotations be made GPU-efficient, or is conversion to quaternions at the boundary always necessary?
\end{enumerate}

%==============================================================================
\section{Conclusion}
%==============================================================================

We have presented Spread-Quadray Rotors as a novel alternative to quaternions for gimbal-lock-free rotation representation. The key innovations are:

\begin{enumerate}
    \item \textbf{Tetrahedral basis geometry}---$109.47°$ angles rather than orthogonal $90°$
    \item \textbf{Spread/cross measures}---rational values for many useful rotations
    \item \textbf{No norm constraint}---freedom in $\mathbb{R}^4$ rather than confinement to $S^3$
    \item \textbf{Explicit Janus polarity}---discrete double-cover rather than implicit $q \equiv -q$
    \item \textbf{Algebraic exactness}---transcendentals deferred until GPU boundary
\end{enumerate}

The Hairy Ball Theorem guarantees that any 3-parameter rotation system will have singularities. Quaternions escape via the hypersphere $S^3$; Quadray rotors escape via $\mathbb{R}^4 \times \mathbb{Z}_2$. Both are valid lifts from the twisted topology of $SO(3)$, but they produce fundamentally different algebraic structures.

Whether Quadray rotors offer practical advantages over quaternions in specific applications remains to be determined through implementation and testing. The theoretical framework, however, is complete and internally consistent.

%==============================================================================
\section*{Acknowledgments}
%==============================================================================

This work draws on:
\begin{itemize}
    \item N.J.\ Wildberger's Rational Trigonometry
    \item Kirby Urner's Quadray coordinate system
    \item Tom Ace's Quadray rotation formulas
    \item R.\ Buckminster Fuller's synergetic geometry
    \item Julian Barbour's Janus Point concept
\end{itemize}

Developed in collaboration with AI assistance (Claude/Anthropic) for mathematical formalization and documentation.

\theendnotes

%==============================================================================
\section*{References}
%==============================================================================

\begin{enumerate}
    \item Brouwer, L.E.J. (1912). ``Über Abbildung von Mannigfaltigkeiten.'' \emph{Mathematische Annalen}.
    \item Wildberger, N.J. (2005). \emph{Divine Proportions: Rational Trigonometry to Universal Geometry}. Wild Egg Books.
    \item Shoemake, K. (1985). ``Animating Rotation with Quaternion Curves.'' \emph{SIGGRAPH}.
    \item Hanson, A.J. (2006). \emph{Visualizing Quaternions}. Morgan Kaufmann.
    \item Fuller, R.B. (1975). \emph{Synergetics: Explorations in the Geometry of Thinking}. Macmillan.
    \item Barbour, J. (2020). \emph{The Janus Point: A New Theory of Time}. Basic Books.
    \item Urner, K. ``Quadray Coordinates: A Logical Alternative.'' \url{http://www.grunch.net/synergetics/quadintro.html}
    \item Ace, T. ``Quadray Coordinates.'' \url{http://minortriad.com/quadray.html}
    \item Thomson, A. (2026). \emph{ARTexplorer: Interactive Tetrahedral Geometry Visualization}. \url{https://arossti.github.io/ARTexplorer/}
\end{enumerate}

\vfill
\begin{center}
\textit{``You cannot comb a hairy ball flat without creating a cowlick.''\\[0.5em]
--- L.E.J.\ Brouwer (1912), as applied to rotation space}
\end{center}

\end{document}
