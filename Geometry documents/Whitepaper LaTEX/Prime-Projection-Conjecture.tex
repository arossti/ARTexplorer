\documentclass[11pt,a4paper]{article}
\usepackage[utf8]{inputenc}
\usepackage[T1]{fontenc}
\usepackage{amsmath,amssymb,amsthm}
\usepackage{geometry}
\usepackage{hyperref}
\usepackage{graphicx}
\usepackage{booktabs}
\usepackage{enumitem}
\usepackage{xcolor}
\usepackage{parskip}
\usepackage{tcolorbox}
\usepackage{listings}
\usepackage{algorithm}
\usepackage{algpseudocode}

\geometry{margin=1in}

\hypersetup{
    colorlinks=true,
    linkcolor=blue,
    urlcolor=blue,
    citecolor=blue
}

\lstset{
    basicstyle=\ttfamily\small,
    breaklines=true,
    frame=single,
    backgroundcolor=\color{gray!10}
}

\newtheorem{conjecture}{Conjecture}
\newtheorem{definition}{Definition}
\newtheorem{observation}{Observation}
\newtheorem{theorem}{Theorem}
\newtheorem{proposition}{Proposition}
\newtheorem{corollary}{Corollary}

\title{The 4D± Prime Projection Conjecture\\
\large Rational-Spread Projections of Higher-Dimensional Polytopes\\
as a Source of Non-Constructible Prime n-Gons\\[0.5em]
\normalsize v5.1 -- February 2026}

\author{Andrew Thomson\\
\small Open Building / ARTexplorer Project\\
\small \href{mailto:andy@openbuilding.ca}{andy@openbuilding.ca}}

\date{February 2026}

\begin{document}

\maketitle

\begin{abstract}
This paper presents the \textbf{4D± Prime Projection Conjecture}: that prime-sided polygons (7, 11, 13, 19...), which are non-constructible in 2D under the Gauss-Wantzel theorem, emerge as rational-spread projections of algebraically-defined polyhedra. We describe the problem of irrational polygon construction, introduce a computational search methodology using rational spread values, and report experimental findings. Key results include the discovery of the \textbf{Symmetry Barrier}---regular polytopes with central symmetry always project to even-sided hull boundaries---and its resolution through \textbf{asymmetric polytopes}, \textbf{compound polyhedra}, and \textbf{single rational polyhedra}.

\textbf{Compound polyhedra results (v4.0):}
\begin{itemize}
    \item \textbf{5-gon}: Truncated tetrahedron at $s=(0, \frac{1}{2}, 0)$ --- Tier~1 rational, $\sqrt{2}$ family
    \item \textbf{7-gon}: TruncTet + DualTet (16v) at $s=(\frac{1}{2}, \frac{1}{2}, \frac{1}{2})$ --- Tier~1, all spreads equal
    \item \textbf{11-gon}: TruncTet + Icosahedron (24v) at $s=(\frac{3}{4}, \frac{1}{4}, \frac{1}{2})$ --- Tier~1, $\sqrt{2}/\sqrt{3}$ family
    \item \textbf{13-gon}: Same compound at $s=(\frac{9}{10}, \frac{24}{25}, \frac{19}{20})$ --- Tier~3, $\sqrt{5}$ family
\end{itemize}

\textbf{Single-polyhedra results (v5.1, Feb 2026):}
\begin{itemize}
    \item \textbf{7-gon}: Geodesic Tet f=2 (10v) at $s=(0, \frac{1}{3}, \frac{1}{3})$ --- $\sqrt{2}/\sqrt{3}$ family, Tier~1
    \item \textbf{11-gon}: Geodesic Tet f=4 (34v) at $s=(\frac{3}{4}, \frac{1}{3}, \frac{1}{3})$ --- Tier~1, $\sqrt{2}/\sqrt{3}$
    \item \textbf{13-gon}: Geodesic Tet f=4 (34v) at $s=(\frac{1}{2}, \frac{3}{4}, \frac{3}{4})$ --- Tier~1, $\sqrt{2}$ only!
\end{itemize}

All four primes emerge at \textbf{exact rational spreads}---not decimal approximations. All four now at Tier~1 denominators $\{2, 3, 4\}$, with rotation matrices expressible entirely in cached $\sqrt{2}$ and $\sqrt{3}$ radicals. The v5.1 results demonstrate that a \textbf{single polyhedron family}---the geodesic tetrahedron at increasing subdivision frequency---produces all three non-constructible primes (7, 11, 13) without any compound engineering. Critically, v5.1 introduces \textbf{degeneracy verification}: hull vertices at 180° interior angles are geometrically meaningless and excluded. All centrally symmetric polyhedra (cuboctahedron, rhombic dodecahedron, geodesic octahedron) produce exclusively degenerate odd-gon projections; only the asymmetric geodesic tetrahedron yields clean primes. Hull vertex counts are verified using \textbf{Path~C exact arithmetic} (\texttt{fractions.Fraction}-based cross products) eliminating all floating-point ambiguity.

We propose the \textbf{Projective Shadow Principle}: \emph{prime polygons are projective shadows of non-prime polyhedra at rational angles}---a combinatorial result about projection geometry fundamentally distinct from compass-and-straightedge constructibility.
\end{abstract}

\tableofcontents
\newpage

%==============================================================================
\section{Introduction: The Problem of Irrational Polygons}
%==============================================================================

\subsection{Classical Polygon Construction}

Constructing a regular $n$-gon requires placing $n$ equally-spaced vertices on a circle. Classically, this involves computing:
\begin{equation}
v_k = \left( R \cos\frac{2\pi k}{n}, \; R \sin\frac{2\pi k}{n} \right), \quad k = 0, 1, \ldots, n-1
\end{equation}

The functions $\sin(\pi/n)$ and $\cos(\pi/n)$ are \textbf{transcendental} for most values of $n$---they cannot be expressed as finite algebraic expressions involving only rationals and radicals.

\subsection{The Gauss-Wantzel Theorem}

\begin{theorem}[Gauss-Wantzel, 1837]
A regular $n$-gon is constructible with compass and straightedge if and only if $n$ is of the form:
\begin{equation}
n = 2^k \times p_1 \times p_2 \times \cdots \times p_m
\end{equation}
where $k \geq 0$ and $p_1, p_2, \ldots, p_m$ are \textbf{distinct Fermat primes} of the form $F_j = 2^{2^j} + 1$.
\end{theorem}

The known Fermat primes are: $3, 5, 17, 257, 65537$.

\begin{corollary}[Constructible n-gons for $n \leq 24$]
The constructible regular polygons with $n \leq 24$ sides are:
\[
n \in \{3, 4, 5, 6, 8, 10, 12, 15, 16, 17, 20, 24\}
\]
\end{corollary}

\begin{table}[h]
\centering
\begin{tabular}{clcc}
\toprule
$n$ & \textbf{Name} & \textbf{Constructible?} & \textbf{Required Algebra} \\
\midrule
3 & Triangle & Yes & Rational \\
4 & Square & Yes & Rational \\
5 & Pentagon & Yes & $\sqrt{5}$ (Golden ratio) \\
6 & Hexagon & Yes & Rational \\
\textbf{7} & \textbf{Heptagon} & \textbf{No} & Cubic (degree 3) \\
8 & Octagon & Yes & $\sqrt{2}$ \\
\textbf{9} & \textbf{Nonagon} & \textbf{No} & Cubic (angle trisection) \\
10 & Decagon & Yes & $\sqrt{5}$ \\
\textbf{11} & \textbf{Hendecagon} & \textbf{No} & Degree 5 \\
12 & Dodecagon & Yes & $\sqrt{3}$ \\
\textbf{13} & \textbf{Tridecagon} & \textbf{No} & Degree 6 \\
\bottomrule
\end{tabular}
\caption{Constructibility of regular polygons. Bold entries are non-constructible primes.}
\end{table}

\subsection{The Accuracy Problem}

In computational geometry, non-constructible polygons require evaluating transcendental functions:
\begin{align}
\sin(\pi/7) &= 0.4338837391175582\ldots \quad \text{(non-terminating, non-repeating)}\\
\cos(\pi/7) &= 0.9009688679024191\ldots \quad \text{(irrational)}
\end{align}

These values are:
\begin{enumerate}
    \item \textbf{Approximations}---floating-point representation introduces error
    \item \textbf{Non-algebraic}---cannot be expressed as roots of rational polynomials
    \item \textbf{Computationally expensive}---transcendental evaluation per vertex
\end{enumerate}

\subsection{The Desire for Rational Construction}

Wildberger's \textbf{Rational Trigonometry} (RT) replaces distance and angle with:
\begin{align}
\text{Quadrance:} \quad Q &= d^2 \quad \text{(distance squared)}\\
\text{Spread:} \quad s &= \sin^2(\theta) \quad \text{(angle measure)}
\end{align}

\begin{observation}[Rational Spreads]
Many useful angles have \textbf{rational} spread values:
\begin{center}
\begin{tabular}{cccc}
\toprule
\textbf{Angle} & $\sin(\theta)$ & $\cos(\theta)$ & \textbf{Spread} $s = \sin^2(\theta)$ \\
\midrule
$30°$ & $1/2$ & $\sqrt{3}/2$ & $\mathbf{1/4}$ \\
$45°$ & $\sqrt{2}/2$ & $\sqrt{2}/2$ & $\mathbf{1/2}$ \\
$60°$ & $\sqrt{3}/2$ & $1/2$ & $\mathbf{3/4}$ \\
$90°$ & $1$ & $0$ & $\mathbf{1}$ \\
\bottomrule
\end{tabular}
\end{center}
\end{observation}

This leads to our central question: \textbf{Can prime $n$-gons emerge from rational operations in a higher-dimensional system?}

%==============================================================================
\section{The Quasicrystal Precedent}
%==============================================================================

\subsection{Penrose Tilings and Forbidden Symmetry}

Penrose tilings exhibit \textbf{5-fold rotational symmetry}---a pattern ``impossible'' in periodic 2D crystal lattices. The \textbf{crystallographic restriction theorem} states that periodic tilings of the plane can only have 2-, 3-, 4-, or 6-fold rotational symmetry.

Yet 5-fold symmetry exists in Penrose tilings. How?

\begin{theorem}[de Bruijn, 1981]
Penrose tilings can be constructed as \textbf{2D projections of 5D hypercubic lattices}. The ``forbidden'' 5-fold symmetry exists naturally in 5D and projects down to what appears impossible in 2D alone.
\end{theorem}

\begin{tcolorbox}[colback=blue!5, colframe=blue!40!black, title=\textbf{The Dimensional Escape}]
The crystallographic restriction limits what can exist \emph{within} 2D. It says nothing about what can \emph{project into} 2D from higher dimensions.

Penrose tilings are not 2D objects that violate the crystallographic restriction---they are \textbf{shadows of 5D structures} that happen to fall on a 2D plane.
\end{tcolorbox}

This suggests a profound possibility: constraints on constructibility in 2D might be circumvented by working in higher dimensions and projecting down.

%==============================================================================
\section{Foundations: Projection Geometry of Polyhedra}
%==============================================================================

Before proceeding to our conjecture, we establish the classical mathematical foundations governing how polyhedra project to 2D. These results---analogous to Euler's $V - E + F = 2$ for polyhedral topology---constrain what regular polygons can emerge from orthographic projection.

\subsection{The Shadow Bound: Euler for Projections}

Just as Euler's formula relates the vertices, edges, and faces of any convex polyhedron, we can state bounds on projection hull counts.

\begin{definition}[Projection Hull]
For a convex polyhedron $P$ with vertex set $V$ and an orthographic projection direction $\vec{d}$, the \textbf{projection hull} $H(P, \vec{d})$ is the convex hull of the projected vertices in 2D. The \textbf{hull count} $h = |H|$ is the number of vertices on the hull boundary.
\end{definition}

\begin{theorem}[Shadow Bound]
For a convex polyhedron with $V$ vertices projected orthographically to 2D:
\begin{equation}
3 \leq h \leq V
\end{equation}
where $h$ is the hull vertex count.
\end{theorem}

\begin{proof}
The lower bound ($h \geq 3$) follows from the convex hull of any finite point set in 2D having at least 3 vertices (a triangle). The upper bound ($h \leq V$) is achieved when all vertices project to the hull boundary---e.g., viewing a prism along its axis.
\end{proof}

\subsection{The Ring Projection Principle}

When projecting along a rotational symmetry axis, vertices organize into ``rings'' that determine the achievable regular polygons.

\begin{definition}[Vertex Ring]
For a polyhedron with $k$-fold rotational symmetry about axis $\vec{a}$, a \textbf{vertex ring} is a set of $k$ vertices related by $360°/k$ rotations about $\vec{a}$, all at equal distance from the axis and equal height along it.
\end{definition}

\begin{theorem}[Ring Stacking Theorem]
When projecting a polyhedron along a $k$-fold symmetry axis, the maximum regular $n$-gon achievable from the projection satisfies:
\begin{equation}
n = k \times m
\end{equation}
where $m$ is the number of complete vertex rings whose angular positions combine to form a regular $(km)$-gon---i.e., rings offset by exactly $360°/(km)$ from each other.
\end{theorem}

\begin{observation}[Ring Alignment]
Two rings of $k$ vertices each, offset by $360°/(2k) = 180°/k$, project to a regular $2k$-gon. Three rings offset by $120°/k$ project to a regular $3k$-gon. The key constraint: rings must be angularly aligned at precise multiples of $360°/(km)$.
\end{observation}

\subsection{Maximum Regular Projections: Platonic Solids}

Applying the Ring Stacking Theorem to the five Platonic solids:

\begin{table}[h]
\centering
\begin{tabular}{lcccccl}
\toprule
\textbf{Solid} & $V$ & \textbf{Best Axis} & \textbf{On Axis} & \textbf{Off Axis} & \textbf{Rings} & \textbf{Max Regular $n$} \\
\midrule
Tetrahedron & 4 & 3-fold (vertex) & 1 & 3 & $1 \times 3$ & \textbf{3} (triangle) \\
Cube & 8 & 3-fold (diagonal) & 2 & 6 & $2 \times 3$ @ 60° & \textbf{6} (hexagon) \\
Octahedron & 6 & 3-fold (face) & 0 & 6 & $2 \times 3$ @ 60° & \textbf{6} (hexagon) \\
Dodecahedron & 20 & 5-fold (face) & 0 & 20$\to$10 hull & $2 \times 5$ @ 36° & \textbf{10} (decagon) \\
Icosahedron & 12 & 5-fold (vertex) & 2 & 10 & $2 \times 5$ @ 36° & \textbf{10} (decagon) \\
\bottomrule
\end{tabular}
\caption{Maximum regular polygon projections for Platonic solids. Dual pairs (cube/octahedron, dodecahedron/icosahedron) share the same maximum due to identical symmetry groups.}
\label{tab:platonic-projections}
\end{table}

\begin{observation}[The Cube's Hexagonal Shadow]
The cube (8 vertices) achieves its maximum regular projection---a \textbf{hexagon}---when viewed along the body diagonal $[1,1,1]$:
\begin{itemize}
    \item 2 vertices lie on the viewing axis (opposite corners at origin and $(1,1,1)$)
    \item 6 vertices form two staggered triangles at heights $1/3$ and $2/3$ along the diagonal
    \item These triangles are offset by 60°, combining to project as a regular hexagon
\end{itemize}
No viewing angle produces a regular polygon with more than 6 vertices from the cube.
\end{observation}

\subsection{Maximum Regular Projections: Archimedean Solids}

The 13 Archimedean solids inherit rotational symmetry from their parent Platonic solids but have more vertices, enabling larger regular projections.

\begin{table}[h]
\centering
\begin{tabular}{lccccl}
\toprule
\textbf{Solid} & $V$ & \textbf{Symmetry} & \textbf{Best Axis} & \textbf{Max Regular $n$} & \textbf{Notes} \\
\midrule
Truncated Tetrahedron & 12 & $T_d$ & 3-fold & \textbf{6} & $2 \times 3$ rings \\
Cuboctahedron & 12 & $O_h$ & 4-fold & \textbf{8} & $2 \times 4$ @ 45° \\
Truncated Cube & 24 & $O_h$ & 3-fold & \textbf{12} & $4 \times 3$ @ 30° \\
Truncated Octahedron & 24 & $O_h$ & 3-fold & \textbf{12} & $4 \times 3$ @ 30° \\
Rhombicuboctahedron & 24 & $O_h$ & 4-fold & \textbf{8} & $2 \times 4$ primary \\
Snub Cube & 24 & $O$ (chiral) & 4-fold & \textbf{8} & No central symmetry! \\
Icosidodecahedron & 30 & $I_h$ & 5-fold & \textbf{10} & $2 \times 5$ primary \\
Truncated Dodecahedron & 60 & $I_h$ & 5-fold & \textbf{20} & $4 \times 5$ @ 18° \\
Truncated Icosahedron & 60 & $I_h$ & 5-fold & \textbf{20} & $4 \times 5$ @ 18° \\
Rhombicosidodecahedron & 60 & $I_h$ & 5-fold & \textbf{10} & $2 \times 5$ primary \\
Great Rhombicosidodecahedron & 120 & $I_h$ & 5-fold & \textbf{20} & $4 \times 5$ @ 18° \\
Snub Dodecahedron & 60 & $I$ (chiral) & 5-fold & \textbf{10} & No central symmetry! \\
\bottomrule
\end{tabular}
\caption{Maximum regular polygon projections for Archimedean solids. Chiral solids (snub cube, snub dodecahedron) lack central symmetry---critical for prime projections.}
\label{tab:archimedean-projections}
\end{table}

\subsection{The Symmetry Bound on Maximum Regular $n$}

\begin{theorem}[Symmetry Ceiling]
For a polyhedron with rotational symmetry group $G$, let $k_{\max}$ be the maximum order of any rotational symmetry axis. Then the maximum regular $n$-gon achievable by projection satisfies:
\begin{equation}
n \leq k_{\max} \times \left\lfloor \frac{V - v_{\text{axis}}}{k_{\max}} \right\rfloor
\end{equation}
where $v_{\text{axis}}$ is the number of vertices lying on the $k_{\max}$-fold axis.
\end{theorem}

\begin{center}
\begin{tabular}{lcccc}
\toprule
\textbf{Symmetry Group} & \textbf{Polyhedra} & $k_{\max}$ & \textbf{Theoretical Max} & \textbf{Achieved} \\
\midrule
Tetrahedral $T_d$ & Tetrahedron, Trunc.\ Tet. & 3 & 6 & 6 \\
Octahedral $O_h$ & Cube, Octahedron, Trunc.\ Cube & 4 & 12 & 12 \\
Icosahedral $I_h$ & Dodeca., Icosa., Trunc.\ Icosa. & 5 & 20 & 20 \\
\bottomrule
\end{tabular}
\end{center}

\subsection{The Regularity vs.\ Hull Count Distinction}

\begin{tcolorbox}[colback=yellow!5, colframe=yellow!50!black, title=\textbf{Critical Distinction}]
The theorems above concern \textbf{regular} $n$-gon projections---where all projected vertices lie on a circle at equal angular spacing.

The \textbf{hull count} (number of vertices on the convex hull boundary) is a weaker property. A projection can have $n$ hull vertices without forming a regular $n$-gon:
\begin{itemize}
    \item \textbf{Regular $n$-gon}: All interior angles equal $(180° - 360°/n)$, all edges equal length
    \item \textbf{$n$-hull}: Convex hull has $n$ vertices---possibly irregular (unequal angles/edges)
\end{itemize}

Our prime projection search targets \textbf{hull counts}, not regular polygons. The 7-hull from truncated tetrahedron has interior angles ranging from 70.5° to 180°---far from the regular heptagon's uniform 128.57° angles.
\end{tcolorbox}

\subsection{The Central Symmetry Barrier (Revisited)}

These foundational results inform the Symmetry Barrier theorem stated later:

\begin{theorem}[Central Symmetry $\Rightarrow$ Even Hull]
\label{thm:central-symmetry}
If a polyhedron has central (inversion) symmetry, then \textbf{all} hull counts from orthographic projection are even.
\end{theorem}

\begin{proof}
For every vertex $v$ on the hull, its antipodal partner $-v$ either:
\begin{enumerate}
    \item Also lies on the hull (adding an even contribution), or
    \item Projects to a point collinear with $v$ and the center (symmetric exclusion/inclusion)
\end{enumerate}
In either case, hull vertices come in pairs, giving even count.
\end{proof}

\begin{corollary}[Central Symmetry Blocks Primes]
Polyhedra with central symmetry (all Platonic solids except tetrahedron, most Archimedean solids) cannot produce prime hull counts greater than 2 from any viewing angle.
\end{corollary}

This is why the \textbf{truncated tetrahedron} (no central symmetry) and \textbf{chiral solids} (snub cube, snub dodecahedron) are the key candidates for prime projection search.

\subsection{Summary: Foundational Bounds}

\begin{tcolorbox}[colback=blue!5, colframe=blue!40!black, title=\textbf{Projection Geometry Foundations}]
\textbf{Established facts constraining our search:}
\begin{enumerate}
    \item \textbf{Shadow Bound}: $3 \leq h \leq V$ for any projection hull
    \item \textbf{Ring Stacking}: Max regular $n$-gon $= k \times m$ (axis order $\times$ aligned rings)
    \item \textbf{Symmetry Ceiling}: Max $n$ bounded by rotation group order
    \item \textbf{Central Symmetry Barrier}: Inversion symmetry $\Rightarrow$ even hull counts only
    \item \textbf{Hull $\neq$ Regular}: Prime hulls exist but are irregular projections
\end{enumerate}

These principles are as fundamental to projection geometry as Euler's $V - E + F = 2$ is to polyhedral topology.
\end{tcolorbox}

%==============================================================================
\section{The 4D± Prime Projection Conjecture}
%==============================================================================

\subsection{Statement of the Conjecture}

\begin{conjecture}[4D± Prime Projection]
Prime $n$-gons (7, 11, 13, 19...) are non-constructible in 2D under the Gauss-Wantzel theorem. However, they may exist as \textbf{rational-spread projections} of 4D± polytope structures in the Quadray coordinate system.
\end{conjecture}

\subsection{ARTexplorer: A Rational Synergetics Geometry Tool}

This research is conducted within \textbf{ARTexplorer} (Algebraic Rational Trigonometry Explorer), an interactive 3D/4D geometry visualization application available at:

\begin{center}
\url{https://arossti.github.io/ARTexplorer/}
\end{center}

ARTexplorer implements three foundational mathematical frameworks:
\begin{enumerate}
    \item \textbf{N.J. Wildberger's Rational Trigonometry}---replacing distance and angle with quadrance ($Q = d^2$) and spread ($s = \sin^2\theta$) to avoid transcendental functions
    \item \textbf{R. Buckminster Fuller's Synergetics}---using tetrahedral geometry as the primary coordinate basis, with the IVM (Isotropic Vector Matrix) as the foundational lattice
    \item \textbf{Kirby Urner's Quadray Coordinates}---four-axis coordinates based on tetrahedron vertices, extended to full 4D with Janus polarity
\end{enumerate}

The application is designed to maintain \textbf{algebraic exactness} as long as possible, deferring floating-point evaluation until the final GPU boundary. This makes it an ideal platform for exploring rational-spread projections of higher-dimensional polytopes.

\subsection{The 4D± Quadray System}

ARTexplorer implements a \textbf{full 4D Quadray coordinate system} with:
\begin{itemize}
    \item \textbf{Four basis vectors} pointing to tetrahedron vertices
    \item \textbf{No zero-sum constraint}---full 4D, not projected to 3D
    \item \textbf{Janus polarity}---discrete $\pm$ state for dimensional sign
    \item \textbf{Spread 8/9} between basis vectors (rational!)
\end{itemize}

The central angle between any two Quadray basis vectors is $109.47°$, with:
\begin{equation}
\cos(109.47°) = -\frac{1}{3}, \quad \sin^2(109.47°) = \frac{8}{9}
\end{equation}

Both values are \textbf{exact rationals}---the tetrahedral geometry is inherently rational-trigonometry compatible.

\subsection{The Search Strategy}

If we construct polyhedra rationally in 4D and project to 2D at carefully chosen \textbf{rational spread rotations}, the visible vertex silhouette might form a prime $n$-gon---even though that $n$-gon is ``non-constructible'' in purely 2D terms.

\subsubsection{Candidate Pre-filtering}

Before searching, we eliminate polyhedra that \textbf{cannot} produce prime hulls. By the Central Symmetry Barrier (Theorem~\ref{thm:central-symmetry}), any polyhedron with inversion symmetry (centrosymmetric) projects only even-sided hulls. This eliminates:
\begin{itemize}
    \item All Platonic solids except tetrahedron (4 vertices---insufficient)
    \item Most Archimedean solids (those with central symmetry)
    \item All prisms and antiprisms with even symmetry
\end{itemize}

Remaining candidates include asymmetric and chiral polyhedra:
\begin{itemize}
    \item \textbf{Truncated tetrahedron} (12 vertices, no inversion symmetry)
    \item \textbf{Snub cube} (24 vertices, chiral)
    \item \textbf{Snub dodecahedron} (60 vertices, chiral)
    \item \textbf{Compound polyhedra} formed from asymmetric combinations
\end{itemize}

\subsubsection{Three-Phase Search Protocol}

\textbf{Phase 1: On-Axis Views} (Exhaustive coverage of special directions)

\begin{table}[h]
\centering
\begin{tabular}{clcl}
\toprule
\textbf{Priority} & \textbf{Axis Type} & \textbf{Views} & \textbf{Rationale} \\
\midrule
1 & Cartesian XYZ & $\pm X, \pm Y, \pm Z$ (6 views) & Standard orthographic \\
2 & Quadray WXYZ & QW, QX, QY, QZ (4 views) & Tetrahedral-aligned \\
\bottomrule
\end{tabular}
\caption{On-axis viewing directions (Phase 1)}
\end{table}

The Quadray axes offer non-obvious viewing angles that Cartesian projections miss. The truncated tetrahedron, for instance, has purely rational Quadray coordinates (permutations of $\{2,1,0,0\}$), making Quadray viewing angles particularly natural for this form.

\textbf{Phase 2: Rational Spread Grid Search (Path A)}

Instead of searching over uniform decimal grids, we search over \textbf{algebraically significant rationals} organized in tiers:

\begin{lstlisting}[language=Python, caption=Path A rational tier search]
# Tier 1 (RT-pure): q in {2, 3, 4}  -> 7 spread values
# Tier 2 (phi):     q in {5, 8, 10}  -> 19 spread values
# Tier 3 (algebraic): q in {6, 9, 12, 16, 20, 25} -> 67 values

for s1 in rational_spreads:      # e.g. 0, 1/4, 1/3, 1/2, 2/3, 3/4, 1
    for s2 in rational_spreads:
        for s3 in rational_spreads:
            R = rotation_from_spreads(s1, s2, s3)
            hull_count = project_and_count(vertices, R)
            if is_prime(hull_count): record(s1, s2, s3, hull_count)
\end{lstlisting}

Tier~1 searches $7^3 = 343$ configurations per polyhedron; Tier~3 searches $67^3 \approx 300{,}000$. This is far smaller than the decimal grid ($101^3 \approx 10^6$) but targets algebraically meaningful spreads whose rotation matrices are expressible in cached radicals ($\sqrt{2}$, $\sqrt{3}$, $\sqrt{5}$).

\textbf{Phase 3: Fine Refinement} (Local optimization)

For each spread tuple $(s_1, s_2, s_3)$ that produces target prime hull:
\begin{itemize}
    \item Search $\pm 0.001$ neighborhood at precision $10^{-4}$
    \item Optimize for \textbf{regularity score}: minimize edge and angle variance
    \item Record ``most regular'' prime hull projection
\end{itemize}

\subsubsection{Pipeline Summary}

\begin{enumerate}
    \item Generating 3D/4D polyhedra with rational vertex coordinates
    \item \textbf{Pre-filtering}: Eliminate centrosymmetric candidates (Central Symmetry Barrier)
    \item \textbf{Phase 1}: Search all 10 on-axis views (6 Cartesian + 4 Quadray)
    \item \textbf{Phase 2}: Grid search rational spreads at increasing precision
    \item Projecting to 2D (orthographic)
    \item Computing the convex hull boundary
    \item Counting hull vertices and checking for prime counts
    \item \textbf{Phase 3}: Refine to maximize regularity score
\end{enumerate}

%==============================================================================
\section{Computational Search Methodology}
%==============================================================================

\subsection{The Prime Projection Search Script}

We developed a Python-based search tool (\texttt{scripts/prime\_search\_streamlined.py}) using RT-pure modules ported directly from JavaScript (\texttt{rt\_math.py}, \texttt{rt\_polyhedra.py}).

\begin{lstlisting}[language=Python, caption=Core search algorithm (pseudocode)]
def search_for_prime(target_prime, rational_tier=0):
    vertices = PRIME_CONFIGS[target_prime]['vertices']()

    if rational_tier > 0:
        spreads = generate_rational_spread_grid(rational_tier)
    else:
        spreads = generate_spread_grid(precision)

    for s1, s2, s3 in product(spreads, repeat=3):
        hull_count = count_hull_vertices(vertices, s1, s2, s3)
        if hull_count == target_prime:
            geometry = compute_hull_geometry(get_hull_points(...))
            record(s1, s2, s3, hull_count, geometry)
\end{lstlisting}

\subsection{Polyhedra Library}

The search includes:

\begin{table}[h]
\centering
\begin{tabular}{lccl}
\toprule
\textbf{Polyhedron} & \textbf{Dimension} & \textbf{Vertices} & \textbf{Notes} \\
\midrule
Tetrahedron & 3D & 4 & Platonic solid \\
Cube & 3D & 8 & Platonic solid \\
Octahedron & 3D & 6 & Platonic solid \\
Icosahedron & 3D & 12 & Platonic solid, $\phi$-based \\
Dodecahedron & 3D & 20 & Platonic solid, $\phi$-based \\
\midrule
Stella Octangula & 3D & 8 & Compound: 2 tetrahedra \\
Truncated Tetrahedron & 3D & 12 & \textbf{No central symmetry!} \\
Snub Cube & 3D & 24 & Chiral (left/right-handed) \\
\midrule
Cuboctahedron & 3D & 12 & Fuller's VE --- \textcolor{red}{degenerate only} \textcolor{green!50!black}{[v5.0]} \\
Rhombic Dodecahedron & 3D & 14 & Dual of cuboctahedron --- \textcolor{red}{degenerate only} \textcolor{green!50!black}{[v5.0]} \\
Geodesic Tet (freq=2) & 3D & 10 & 7-gon (clean) \textcolor{green!50!black}{[v5.0]} \\
Geodesic Tet (freq=4) & 3D & 34 & \textbf{11, 13-gon (clean)} \textcolor{green!50!black}{[v5.1]} \\
Geodesic Oct (freq=2) & 3D & 18 & \textcolor{red}{degenerate only} \textcolor{green!50!black}{[v5.0]} \\
\midrule
Tesseract & 4D & 16 & 4D hypercube \\
24-cell & 4D & 24 & Unique to 4D \\
600-cell & 4D & 216 & 4D analogue of icosahedron \\
\bottomrule
\end{tabular}
\caption{Polyhedra included in the search. v5.0 additions are single rational polyhedra in the $\sqrt{2}$ family.}
\end{table}

\subsection{Rational Spread Rotation}

Rotation matrices are constructed from spread values using the relationship:
\begin{align}
\sin(\theta) &= \sqrt{s} \\
\cos(\theta) &= \sqrt{1 - s}
\end{align}

For 3D polyhedra, we use three rotation parameters (ZYX Euler convention).
For 4D polytopes, we use six rotation parameters (one per rotation plane: XY, XZ, XW, YZ, YW, ZW).

\subsection{Search Parameters}

\begin{table}[h]
\centering
\begin{tabular}{lc}
\toprule
\textbf{Parameter} & \textbf{Value} \\
\midrule
Spread precision & 2 decimal places (101 values: 0.00--1.00) \\
3D rotation configurations & $101^3 \approx 1.03 \times 10^6$ per polyhedron \\
4D rotation configurations & $101^6 \approx 1.06 \times 10^{12}$ (sampled) \\
Target primes & 7, 11, 13, 17, 19, 23, 29, 31 \\
Vertex tolerance & $10^{-6}$ (for deduplication) \\
\bottomrule
\end{tabular}
\caption{Search configuration}
\end{table}

%==============================================================================
\section{Experimental Findings}
%==============================================================================

\subsection{The Symmetry Barrier}

\begin{observation}[Even Hull Count Phenomenon]
Initial experiments on regular polytopes revealed that \textbf{all observed hull counts are even}:
\begin{center}
\begin{tabular}{lcc}
\toprule
\textbf{Polytope} & \textbf{Vertices} & \textbf{Observed Hull Counts} \\
\midrule
Dodecahedron & 20 & 10, 12 only \\
600-cell & 216 & 12, 14, 16, 18, 20, 22, 24, 26 \\
\bottomrule
\end{tabular}
\end{center}
\end{observation}

\begin{theorem}[Symmetry Barrier]
Regular polytopes with \textbf{inversion symmetry} (point reflection through center) always project to 2D with even hull vertex counts.
\end{theorem}

\begin{proof}[Proof sketch]
For a polytope with inversion symmetry, every vertex $v$ has a paired vertex $-v$ at the diametrically opposite position. Under orthographic projection to 2D:
\begin{enumerate}
    \item If $v$ is on the convex hull boundary, then $-v$ is also on the hull boundary (or the interior, but symmetrically)
    \item The hull boundary consists of vertex pairs $(v_i, -v_i)$
    \item Therefore, the hull vertex count is even
\end{enumerate}
\end{proof}

\begin{corollary}
Prime $n$-gon projections (for primes $> 2$) cannot arise from centrally symmetric polytopes.
\end{corollary}

\subsection{Breaking the Symmetry Barrier}

The key insight: \textbf{asymmetric polytopes} can break the even-hull constraint.

\begin{definition}[Asymmetric Polytope]
A polytope is \textbf{asymmetric} if it lacks inversion symmetry---there is no center point $c$ such that for every vertex $v$, the point $2c - v$ is also a vertex.
\end{definition}

Candidates for asymmetric polyhedra:
\begin{enumerate}
    \item \textbf{Truncated tetrahedron}---Archimedean solid, 12 vertices, NO central symmetry
    \item \textbf{Snub cube}---chiral Archimedean solid, 24 vertices, NO central symmetry
    \item \textbf{Compound of 5 tetrahedra}---20 vertices, chiral
\end{enumerate}

\subsection{Breakthrough: The Truncated Tetrahedron}

Exhaustive search on the truncated tetrahedron (12 vertices) with precision 0.05 (21 values per axis, 9,261 total configurations) yielded:

\begin{table}[h]
\centering
\begin{tabular}{cccc}
\toprule
\textbf{Hull Count} & \textbf{Frequency} & \textbf{Percentage} & \textbf{Type} \\
\midrule
\textbf{5-gon} & 15 & 0.2\% & \textcolor{green!50!black}{\textbf{PRIME (Fermat)}} \\
6-gon & 42 & 0.5\% & Even \\
\textbf{7-gon} & 6 & 0.1\% & \textcolor{green!50!black}{\textbf{PRIME (Non-constructible!)}} \\
8-gon & 5,145 & 55.6\% & Even \\
\textbf{9-gon} & 4,053 & 43.8\% & Odd (cubic-algebraic) \\
\bottomrule
\end{tabular}
\caption{Hull count distribution for truncated tetrahedron projections}
\end{table}

\begin{tcolorbox}[colback=green!5, colframe=green!40!black, title=\textbf{Key Discovery: 7-gon Projection at Rational Spreads}]
The \textbf{heptagon (7-gon)} is NOT compass-constructible under Gauss-Wantzel (requires solving a cubic equation). Yet it emerges as a rational-spread projection of the TruncTet+DualTet compound:

\begin{center}
\begin{tabular}{cccc}
\toprule
\textbf{Prime} & \textbf{Rational Spreads} & \textbf{Polyhedron} & \textbf{Tier} \\
\midrule
5-gon & $(0, \frac{1}{2}, 0)$ & Truncated Tetrahedron (12v) & 1 \\
\textbf{7-gon} & $(\frac{1}{2}, \frac{1}{2}, \frac{1}{2})$ & TruncTet+DualTet (16v) & 1 \\
\bottomrule
\end{tabular}
\end{center}

The 7-gon at $s=(\frac{1}{2}, \frac{1}{2}, \frac{1}{2})$---all three spreads equal and rational---produces a \textbf{heptagonal hull} with regularity 0.861. The rotation matrix is pure $\sqrt{2}/2$ throughout, cached in \texttt{RT.PureRadicals}.
\end{tcolorbox}

\subsection{Interpretation}

The 7-gon projection arises because:
\begin{enumerate}
    \item The truncated tetrahedron lacks inversion symmetry---odd hull counts are possible
    \item At specific rational spread viewing angles, exactly 7 vertices fall on the convex hull boundary
    \item The remaining 5 vertices project to the interior of the hull
    \item The result is a heptagonal silhouette from a 3D solid
\end{enumerate}

This does not violate Gauss-Wantzel: we are not \emph{constructing} a heptagon in 2D. We are \emph{projecting} a 3D object that happens to have 7 visible boundary vertices at this viewing angle.

\subsection{★ BREAKTHROUGH: 11-gon and 13-gon Projections (Feb 2026)}

The search for higher primes required combining asymmetric polyhedra into \textbf{compound structures}. The key insight: combining the truncated tetrahedron (no central symmetry) with other polyhedra produces hulls with more than 12 vertices, enabling prime counts beyond the single-polyhedron limit.

\begin{tcolorbox}[colback=green!5, colframe=green!40!black, title=\textbf{MAJOR DISCOVERY: 11-gon and 13-gon at Rational Spreads}]
\textbf{Compound: Truncated Tetrahedron + Icosahedron} (24 vertices)

\begin{center}
\begin{tabular}{ccccl}
\toprule
\textbf{Prime} & \textbf{Rational Spreads} & \textbf{Tier} & \textbf{Radical Family} & \textbf{Algebraic Req.} \\
\midrule
\textbf{11-gon} & $(\frac{3}{4}, \frac{1}{4}, \frac{1}{2})$ & 1 & $\sqrt{2}, \sqrt{3}$ & Quintic (degree 5) \\
\textbf{13-gon} & $(\frac{9}{10}, \frac{24}{25}, \frac{19}{20})$ & 3 & $\sqrt{5}$ family & Sextic (degree 6) \\
\bottomrule
\end{tabular}
\end{center}

\textbf{Significance}: The 11-gon uses \textbf{Tier~1 rational spreads}---denominators $\{2, 4\}$ only---the same $\sqrt{2}/\sqrt{3}$ radical family as the 5-gon and 7-gon. A non-constructible prime polygon emerges from the simplest possible algebraic spreads.

\textbf{Note (v4.0):} Rational spreads found via \texttt{--rational TIER} search in \texttt{prime\_search\_streamlined.py}. Results carry exact rational labels (e.g.\ ``3/4'') and work directly in JavaScript \texttt{PROJECTION\_PRESETS}.
\end{tcolorbox}

\subsubsection{Compound Configuration}

The compound is constructed by:
\begin{enumerate}
    \item Taking the truncated tetrahedron (12 vertices, no central symmetry)
    \item Adding the icosahedron (12 vertices, with central symmetry)
    \item Total: 24 vertices with broken combined symmetry
\end{enumerate}

The relative rotation between the two polyhedra is $s_{\text{rel}} = 0$ (aligned), with only the viewing angle varying.

\subsubsection{Hull Count Distribution for Compound}

\begin{table}[h]
\centering
\begin{tabular}{cccc}
\toprule
\textbf{Hull Count} & \textbf{Configurations} & \textbf{Type} & \textbf{Notes} \\
\midrule
7-hull & 11 & \textcolor{green!50!black}{\textbf{PRIME}} & From truncated tet component \\
8-hull & 1,419 & Even & \\
9-hull & 748 & Odd (cubic) & \\
10-hull & 1,309 & Even & \\
\textbf{11-hull} & \textbf{495} & \textcolor{green!50!black}{\textbf{PRIME (quintic!)}} & Hendecagon \\
12-hull & 341 & Even & \\
\textbf{13-hull} & \textbf{33} & \textcolor{green!50!black}{\textbf{PRIME (sextic!)}} & Tridecagon \\
\bottomrule
\end{tabular}
\caption{Hull count distribution for truncated tetrahedron + icosahedron compound}
\end{table}

\subsubsection{Geometry of the 11-gon Projection}

At rational spreads $s = (\frac{3}{4}, \frac{1}{4}, \frac{1}{2})$, the 24-vertex compound projects to an 11-vertex convex hull:

\begin{itemize}
    \item \textbf{Regularity score}: 0.490 (Tier~1 rational)
    \item \textbf{Angle variance}: 11.00°
    \item \textbf{Edge variance}: 37.03\%
    \item \textbf{Rational tier}: 1---denominators $\{2, 4\}$ only
    \item \textbf{Radical family}: $\sqrt{1/4} = 1/2$, $\sqrt{3/4} = \sqrt{3}/2$, $\sqrt{1/2} = \sqrt{2}/2$---all cached
\end{itemize}

Alternative Tier~3 result: $s = (\frac{1}{2}, \frac{9}{20}, \frac{1}{3})$ achieves regularity 0.504, nearly matching the decimal grid search best.

\subsubsection{Geometry of the 13-gon Projection}

At rational spreads $s = (\frac{9}{10}, \frac{24}{25}, \frac{19}{20})$, the compound projects to a 13-vertex convex hull:

\begin{itemize}
    \item \textbf{Regularity score}: 0.346 (Tier~3 rational)
    \item \textbf{Angle variance}: 15.22°
    \item \textbf{Edge variance}: 43.46\%
    \item \textbf{Rational tier}: 3---denominators $\{10, 20, 25\}$, $\sqrt{5}$ family
\end{itemize}

The 13-gon does \emph{not} appear at Tier~1 rationals---this is itself a meaningful result, suggesting it inherently requires the golden ratio radical family.

\subsubsection{Alternative Compound: 2× Truncated Tetrahedron}

Combining two truncated tetrahedra (24 vertices total) with relative rotation also yields primes:

\begin{center}
\begin{tabular}{cccc}
\toprule
\textbf{Prime} & \textbf{Relative Spread} & \textbf{View Spreads} & \textbf{Configurations} \\
\midrule
11-gon & 0.1 & $(0, 0, 0.2)$ & 1,368 \\
13-gon & 0.1 & $(0, 0, 1.0)$ & 418 \\
\bottomrule
\end{tabular}
\end{center}

This compound also produces 15-hull and 16-hull configurations, suggesting paths to even higher primes with larger compounds.

\subsection{★ Single-Polyhedra Prime Projections (v5.0--5.1)}

A more elegant result than compound polyhedra: finding prime polygon hulls from \textbf{single, non-compound rational polyhedra}. No combining two polyhedra to break symmetry---just one algebraically-defined solid, viewed at a rational spread angle, producing a prime hull. Hull counts verified with \textbf{Path~C exact arithmetic} (Section~\ref{sec:pathc}).

\subsubsection{Search Polyhedra and Degeneracy Discovery}

The initial v5.0 search tested four candidates (cuboctahedron, rhombic dodecahedron, geodesic tetrahedron, geodesic octahedron). Subsequent \textbf{degeneracy verification} (v5.1) revealed a critical insight: hull vertices at 180° interior angles are geometrically meaningless---the exact cross product is positive but $\sim 10^{-17}$, indicating the vertex lies exactly on the line between its neighbors.

\begin{table}[h]
\centering
\begin{tabular}{lccccc}
\toprule
\textbf{Polyhedron} & \textbf{V} & \textbf{Radical} & \textbf{Central Sym.} & \textbf{Odd Hull?} & \textbf{Status} \\
\midrule
Cuboctahedron & 12 & $\sqrt{2}$ & Yes & 7 & \textcolor{red}{DEGENERATE (180°)} \\
Rhombic Dodec & 14 & $\sqrt{2}$ & Yes & 7 & \textcolor{red}{DEGENERATE (180°)} \\
Geodesic Oct f=2 & 18 & $\sqrt{2}$ & Yes & 11 & \textcolor{red}{DEGENERATE (180°)} \\
Geodesic Tet f=2 & 10 & $\sqrt{2}, \sqrt{3}$ & \textbf{No} & 7 & \textcolor{green!50!black}{\textbf{CLEAN (152°)}} \\
Geodesic Tet f=4 & 34 & $\sqrt{2}, \sqrt{3}$ & \textbf{No} & 11, 13 & \textcolor{green!50!black}{\textbf{CLEAN (170°, 173°)}} \\
\bottomrule
\end{tabular}
\caption{Single-polyhedra search results with degeneracy verification (v5.1). ALL centrally symmetric polyhedra produce exclusively degenerate odd-gon projections. Only the asymmetric geodesic tetrahedron yields clean primes.}
\end{table}

\begin{tcolorbox}[colback=red!5, colframe=red!40!black, title=\textbf{THE CENTRAL SYMMETRY BARRIER (Strengthened, v5.1)}]
Centrally symmetric polyhedra (cuboctahedron, rhombic dodecahedron, geodesic octahedron) produce odd hull counts that are \emph{exclusively degenerate}---one or more hull vertices lie at exactly 180° interior angles. The hull count is technically correct under Path~C exact arithmetic (the cross product is positive), but the geometry is meaningless: the ``vertex'' is collinear with its neighbors.

\textbf{Only asymmetric polyhedra} (those lacking central inversion symmetry) produce \textbf{non-degenerate} odd hull counts. The geodesic tetrahedron inherits the base tetrahedron's lack of central symmetry, which is precisely what allows odd hull counts to emerge cleanly.
\end{tcolorbox}

\begin{tcolorbox}[colback=green!5, colframe=green!40!black, title=\textbf{VERIFIED SINGLE-POLYHEDRA PRIME PROJECTIONS (v5.1)}]

\begin{center}
\begin{tabular}{clcccccl}
\toprule
\textbf{Prime} & \textbf{Source} & \textbf{V} & \textbf{Spreads} & \textbf{Tier} & \textbf{Reg.} & \textbf{Max$\angle$} & \textbf{Radical} \\
\midrule
\textbf{7} & Geodesic Tet f=2 & 10 & $(0, \frac{1}{3}, \frac{1}{3})$ & 1 & 0.419 & 152° & $\sqrt{2},\sqrt{3}$ \\
\textbf{11} & Geodesic Tet f=4 & 34 & $(\frac{3}{4}, \frac{1}{3}, \frac{1}{3})$ & 1 & 0.432 & 170° & $\sqrt{2},\sqrt{3}$ \\
\textbf{13} & Geodesic Tet f=4 & 34 & $(\frac{1}{2}, \frac{3}{4}, \frac{3}{4})$ & 1 & 0.448 & 173° & $\sqrt{2}$ \\
\textbf{13} & Geodesic Tet f=4 & 34 & $(\frac{1}{3}, \frac{2}{3}, \frac{3}{4})$ & 1 & 0.443 & 171° & $\sqrt{2},\sqrt{3}$ \\
\bottomrule
\end{tabular}
\end{center}

\textbf{Significance}:
\begin{enumerate}
    \item \textbf{First single-polyhedra 13-gon}: The geodesic tet at freq=4 (34v) produces a 13-gon at Tier~1 with reg=0.448, \emph{beating} the compound result (Tier~3, reg=0.346, borderline 179°) in every metric. The 13-gon uses only $\sqrt{2}$ radicals---same family as the 5-gon and 7-gon.
    \item \textbf{All non-constructible primes from ONE polyhedron family}: The geodesic tetrahedron at frequencies 2 and 4 produces 7, 11, \emph{and} 13 at Tier~1 rational spreads. No compounds, no golden ratio, no icosahedral geometry.
    \item \textbf{7-gon from 10 vertices}: The geodesic tetrahedron has only 10 vertices---the fewest of any source---yet produces heptagon hulls at Tier~1. Finding primes from non-primes: 10 is even, yet the hull is prime.
    \item \textbf{All four primes now at Tier~1}: With the single-poly 13-gon at $s=(\frac{1}{2}, \frac{3}{4}, \frac{3}{4})$, all four target primes (5, 7, 11, 13) emerge at Tier~1 denominators $\{2, 3, 4\}$.
\end{enumerate}
\end{tcolorbox}

\subsubsection{The Elegance Argument}

A single, rationally-constructed polyhedron viewed at a rational angle producing a prime polygon is far more compelling than a compound polyhedron engineered to break symmetry. We propose an \textbf{elegance ranking} for prime projection sources:

\begin{enumerate}
    \item Single polyhedra at Tier~1 $>$ Single at Tier~2+ $>$ Compound at Tier~1 $>$ Compound at Tier~2+
    \item Fewer vertices $>$ more vertices (7-gon from 10v geodesic tet is more elegant than from 16v compound)
    \item $\sqrt{2}$-only $> \sqrt{2}+\sqrt{3} > \sqrt{5}/\phi$ (simpler radical family)
    \item \textbf{Non-degenerate only} --- hull vertices at 180° interior angles disqualify results
\end{enumerate}

By this ranking, the geodesic tetrahedron's 7-gon (10 vertices, Tier~1, $\sqrt{2}/\sqrt{3}$) is the most elegant heptagon source despite its lower regularity (0.419 vs 0.861 for the compound). The geodesic tetrahedron f=4's 13-gon ($\sqrt{2}$ only, Tier~1, clean) decisively outranks the compound 13-gon (Tier~3, $\sqrt{5}$, borderline 179°).

\subsubsection{Geodesic Subdivision}

The geodesic polyhedra use Fuller's frequency notation for triangular face subdivision:

\begin{center}
\begin{tabular}{lccc}
\toprule
\textbf{Base} & \textbf{freq=2} & \textbf{freq=3} & \textbf{freq=4} \\
\midrule
Tetrahedron (4v, 4f) & 10v & 20v & 34v \\
Octahedron (6v, 8f) & 18v & 38v & 66v \\
\bottomrule
\end{tabular}
\end{center}

Each edge is divided into $f$ segments, creating $f^2$ sub-triangles per face. All subdivision vertices are projected onto the circumscribing sphere, producing a geodesic dome. Higher frequencies provide more hull candidates and may reveal 5-gon and 13-gon from single polyhedra.

%==============================================================================
\section{Compound Polyhedra and Relative Rotations}
%==============================================================================

\subsection{Dynamic Compound Generation}

Beyond fixed asymmetric polyhedra, we can create \textbf{compound polyhedra} by combining two polyhedra with a relative rotation:

\begin{lstlisting}[language=Python, caption=Compound generation with relative rotation]
def create_compound(poly1, poly2, relative_spread):
    verts1 = get_vertices(poly1)
    verts2 = get_vertices(poly2)

    # Rotate second polyhedron by relative_spread
    R = rotation_matrix_from_spread(relative_spread)
    verts2_rotated = verts2 @ R.T

    # Combine vertices
    return np.vstack([verts1, verts2_rotated])
\end{lstlisting}

This allows searching a 4-dimensional parameter space: 3 viewing angles + 1 relative rotation angle.

\subsection{Stella Octangula Variations}

The Stella Octangula (compound of two tetrahedra) is a classic dual compound. Varying the relative rotation between the two tetrahedra creates a family of configurations that may yield different prime projections.

\subsection{Search for Higher Primes}

Finding 11-gon, 13-gon, 17-gon, etc. requires:
\begin{enumerate}
    \item Polyhedra with more vertices (to have enough boundary candidates)
    \item More extensive searches (finer precision, more configurations)
    \item Potentially 4D polytopes with asymmetric structure
\end{enumerate}

\textbf{Update (v4.0):} The compound approach proved successful---TruncTet+DualTet yields 7-gon, TruncTet+Icosa yields 11-gon and 13-gon. All at exact rational spreads. The search for 17-gon and higher primes continues with larger compounds and the rational tier framework.

%==============================================================================
\section{Implementation Details}
%==============================================================================

\subsection{Script Usage}

\begin{lstlisting}[language=bash, caption=Command-line examples]
cd ARTexplorer/scripts

# Tier 1 rational search (RT-pure: denominators 2,3,4)
python prime_search_streamlined.py --primes 5,7,11 --rational 1 -v

# Tier 3 rational search (algebraic: 67 spread values)
python prime_search_streamlined.py --primes 11,13 --rational 3 --top 10 -v

# Legacy decimal grid search (precision 2 = 101 values per axis)
python prime_search_streamlined.py --primes 5,7 --precision 2 --verify

# Full rational search with JSON output
python prime_search_streamlined.py --primes 5,7,11,13 --rational 3 \
    --output rational_results.json -v

# v5.0: Multi-polyhedra search with exact hull (Path C)
python prime_search_streamlined.py --poly cuboctahedron,rhombicdodec,geodtet,geodoct \
    --primes 5,7,11,13 --rational 1 --exact -v

# v5.0: Geodesic search at higher frequency
python prime_search_streamlined.py --poly geodtet,geodoct --freq 3 \
    --primes 5,7,11,13 --rational 1 --exact -v
\end{lstlisting}

\subsection{Output Format}

Results are saved as JSON for ARTexplorer visualization:

\begin{lstlisting}[language=json, caption=Example output format]
{
  "metadata": {
    "timestamp": "2026-02-06T12:00:00",
    "precision": 2,
    "target_primes": [7, 11, 13]
  },
  "findings": [
    {
      "polyhedron": "truncated_tetrahedron",
      "spreads": [0.15, 0.0, 0.5],
      "hull_vertices": 7,
      "projected_points": [[x1, y1], [x2, y2], ...]
    }
  ]
}
\end{lstlisting}

\subsection{Performance}

The search is parallelized across CPU cores using Python's \texttt{multiprocessing}:
\begin{itemize}
    \item 3D polyhedra: $\sim$1 million configurations per minute (8-core machine)
    \item 4D polytopes: Coarse precision (0.1) required due to 6-parameter space
    \item Compound pairs: 4-parameter space, moderate search times
\end{itemize}

%==============================================================================
\section{Reproducibility Resources}
%==============================================================================

All code, data, and results are available in the ARTexplorer Git repository for peer review and reproducibility.

\subsection{Repository Structure}

\begin{lstlisting}[language=bash, caption=Key files and directories]
ARTexplorer/
|-- scripts/
|   |-- rt_math.py                        # RT functions ported from rt-math.js
|   |-- rt_polyhedra.py                   # Exact vertex copies from rt-polyhedra.js
|   |-- prime_search_streamlined.py       # Unified search with --rational TIER
|
|-- Geometry documents/
|   |-- Prime-Projection-Conjecture.tex   # This whitepaper
|   |-- Rational-Primes.md               # Path A/B/C rationalization strategy
|   |-- Polygon-Rationalize.md            # RT polygon theory reference
|
|-- modules/
|   |-- rt-math.js                        # Source of truth: RT functions
|   |-- rt-polyhedra.js                   # Source of truth: Polyhedra definitions
|   |-- rt-projections.js                 # Projection visualization
|   |-- rt-prime-cuts.js                  # PROJECTION_PRESETS registry
\end{lstlisting}

\subsection{Running the Search}

\begin{lstlisting}[language=bash, caption=Reproducing the breakthrough results]
# Clone the repository
git clone https://github.com/arossti/ARTexplorer.git
cd ARTexplorer/scripts

# Reproduce Tier 1 rational results (5, 7, 11-gon)
python prime_search_streamlined.py --primes 5,7,11 --rational 1 --verify -v

# Reproduce Tier 3 rational results (13-gon)
python prime_search_streamlined.py --primes 13 --rational 3 --top 5 --verify -v

# Full rational search for all primes
python prime_search_streamlined.py --primes 5,7,11,13 --rational 3 \
    --output rational_verified.json --verify -v

# v5.0: Reproduce single-polyhedra results with exact hull
python prime_search_streamlined.py --poly cuboctahedron,rhombicdodec,geodtet,geodoct \
    --primes 5,7,11,13 --rational 1 --exact -v
\end{lstlisting}

\textbf{Key insight (Path A):} The \texttt{--rational TIER} flag searches over algebraically significant rationals with small denominators. Results carry exact rational labels (e.g.\ ``3/4'') and work directly in JavaScript \texttt{PROJECTION\_PRESETS}. Python and JavaScript use identical vertex definitions ported from \texttt{rt-math.js} and \texttt{rt-polyhedra.js}.

\textbf{Key insight (Path C):} The \texttt{--exact} flag activates \texttt{Fraction}-based cross products in the Graham scan. Combined with \texttt{--poly}, this enables deterministic hull counting for any polyhedron. The \texttt{--freq N} flag controls geodesic subdivision frequency.

\subsection{Results JSON Schema}

Each results file contains:

\begin{lstlisting}[language=json, caption=JSON output schema (v4.0 rational mode)]
{
  "metadata": {
    "timestamp": "ISO-8601 timestamp",
    "precision": 2,
    "rational_tier": 3,
    "mode": "rational_tier_3",
    "source": "prime_search_streamlined.py"
  },
  "primes": {
    "11": {
      "config": { "name": "hendecagon", "polyhedra": "trunc_tet_plus_icosa" },
      "results": [
        {
          "s1": 0.75, "s2": 0.25, "s3": 0.5,
          "s1_rational": "3/4", "s2_rational": "1/4", "s3_rational": "1/2",
          "hull_count": 11,
          "regularity_score": 0.4901,
          "angle_variance": 11.00,
          "edge_variance": 37.03
        }
      ]
    }
  }
}
\end{lstlisting}

\subsection{Verified Findings Summary}

\begin{table}[h]
\centering
\begin{tabular}{cllclccl}
\toprule
\textbf{Prime} & \textbf{Source} & \textbf{Type} & \textbf{V} & \textbf{Rational Spreads} & \textbf{Tier} & \textbf{Max$\angle$} & \textbf{Reg.} \\
\midrule
5 & Truncated Tet & Single & 12 & $(0, \frac{1}{2}, 0)$ & 1 & 144° & 0.423 \\
\midrule
7 & TruncTet + DualTet & Compound & 16 & $(\frac{1}{2}, \frac{1}{2}, \frac{1}{2})$ & 1 & 136° & 0.861 \\
7 & Geodesic Tet f=2 & \textbf{Single} & 10 & $(0, \frac{1}{3}, \frac{1}{3})$ & 1 & 152° & 0.419 \\
\midrule
11 & TruncTet + Icosa & Compound & 24 & $(\frac{3}{4}, \frac{1}{4}, \frac{1}{2})$ & 1 & 165° & 0.490 \\
11 & Geodesic Tet f=4 & \textbf{Single} & 34 & $(\frac{3}{4}, \frac{1}{3}, \frac{1}{3})$ & 1 & 170° & 0.432 \\
\midrule
13 & Geodesic Tet f=4 & \textbf{Single} & 34 & $(\frac{1}{2}, \frac{3}{4}, \frac{3}{4})$ & 1 & 173° & 0.448 \\
13 & TruncTet + Icosa & Compound & 24 & $(\frac{9}{10}, \frac{24}{25}, \frac{19}{20})$ & 3 & 179° & 0.346 \\
\bottomrule
\end{tabular}
\caption{Prime projection leaderboard (v5.1, degeneracy-verified): compound and single-polyhedra results at rational spreads. \textbf{Bold} type = single polyhedra. All results verified non-degenerate (max interior angle $< 179°$) via Path~C exact arithmetic. Degenerate results (cuboctahedron, rhombic dodecahedron, geodesic octahedron --- all 180°) are excluded.}
\end{table}

\subsection{Interactive Visualization}

The ARTexplorer web application provides interactive verification:

\begin{enumerate}
    \item \textbf{Live demo}: \url{https://arossti.github.io/ARTexplorer/}
    \item Enable ``Quadray Truncated Tetrahedron'' in Section 3B
    \item Use ``Prime Projection Views'' presets to verify 5-gon and 7-gon
    \item Inspect console output for hull vertex coordinates
\end{enumerate}

\subsection{Citation}

\begin{lstlisting}[caption=BibTeX citation]
@misc{thomson2026prime,
  author = {Thomson, Andrew},
  title = {The 4D± Prime Projection Conjecture: Rational-Spread
           Projections as a Source of Non-Constructible Prime n-Gons},
  year = {2026},
  howpublished = {\url{https://github.com/arossti/ARTexplorer}},
  note = {v5.1, February 2026. DOI: 10.13140/RG.2.2.23043.98089}
}
\end{lstlisting}

%==============================================================================
\section{Discussion}
%==============================================================================

\subsection{Path C: Exact Arithmetic Verification}
\label{sec:pathc}

A persistent challenge in hull-based prime search is floating-point ambiguity. The Graham scan convex hull uses the cross product $(\vec{a} - \vec{o}) \times (\vec{b} - \vec{o})$ to determine collinearity: positive means left turn (keep), negative means right turn (pop), zero means collinear (exclude). In IEEE 754 arithmetic, ``nearly zero'' cross products produce platform-dependent results.

\textbf{Path~C resolves this definitively.} Python's \texttt{fractions.Fraction(float)} converts each Float64 projected coordinate to its \emph{exact} rational representation (the unique rational $p/q$ that IEEE 754 encodes), then computes cross products with arbitrary-precision integer arithmetic:

\begin{lstlisting}[language=Python, caption=Path C exact cross product (from \texttt{rt\_math.py})]
from fractions import Fraction

frac_pts = [(Fraction(p[0]), Fraction(p[1])) for p in projected_2d]

# In Graham scan:
cross = (a[0] - o[0]) * (b[1] - o[1]) - (a[1] - o[1]) * (b[0] - o[0])
# cross is a Fraction -- exactly zero or not. No epsilon, no ambiguity.
if cross <= 0:
    hull.pop()  # Deterministic exclusion
\end{lstlisting}

When the search reports hull=7 at $s=1/2$ with \texttt{--exact}, the cross products are computed with unlimited precision. Points that are exactly collinear in Float64 produce cross=0 (exactly), and are deterministically excluded. The hull count is provably correct within the Float64 projection model.

All single-polyhedra results in this paper are verified with Path~C exact arithmetic and degeneracy checking (v5.1: hull vertices at 180° interior angles excluded).

\subsection{Relationship to Gauss-Wantzel}

Our findings do not contradict the Gauss-Wantzel theorem. The theorem states what can be \emph{constructed} in 2D with compass and straightedge. We are not constructing---we are \emph{projecting}.

The distinction:
\begin{itemize}
    \item \textbf{Construction}: Creating a shape from first principles using geometric operations
    \item \textbf{Projection}: Casting a shadow of an existing higher-dimensional object
\end{itemize}

A heptagonal shadow of a truncated tetrahedron is not a constructed heptagon. It is a \textbf{view} of a 3D object that happens to have 7-fold boundary structure at that angle.

\subsection{Relationship to Penrose Tilings}

Penrose tilings emerge as 2D projections of 5D hypercubic lattices. The ``impossible'' 5-fold symmetry exists in 5D and projects down.

Our prime projections follow the same principle: the truncated tetrahedron is a 3D object with specific symmetry properties. When viewed from the right angle, 7 of its 12 vertices form the visible boundary---a ``7-fold structure'' that could not be constructed in 2D alone.

\subsection{Implications for Rational Trigonometry}

The viewing angles that produce prime projections are specified in terms of \textbf{exact rational spreads}:
\begin{equation}
\text{7-gon:} \quad s_1 = s_2 = s_3 = \frac{1}{2} \qquad \text{11-gon:} \quad s = \left(\frac{3}{4}, \frac{1}{4}, \frac{1}{2}\right)
\end{equation}

These are not decimal approximations---they are exact rationals with small denominators from $\{2, 3, 4\}$. The rotation matrices require only $\sqrt{2}/2$ and $\sqrt{3}/2$, both cached in \texttt{RT.PureRadicals}. No transcendental function is evaluated anywhere in the pipeline from vertex definition through projection.

This is a strong validation of Wildberger's program: the entire prime polygon discovery---vertices, rotations, projections---operates within the algebraic framework of Rational Trigonometry.

\subsection{Toward a General Theorem: The Projective Shadow Principle}
\label{sec:psp}

The accumulating evidence---verified 7-gon, 11-gon, and 13-gon from a single polyhedron family (geodesic tetrahedron) at Tier~1 rational spreads---suggests a general result rather than isolated numerical coincidence.

\begin{conjecture}[Projective Shadow Principle]
For any convex polyhedron $P$ with $V$ vertices at algebraic coordinates, the set of rotations $R \in SO(3)$ for which the 2D projection of $R(P)$ has exactly $k$ convex hull vertices is either empty or has positive measure. In particular, for sufficiently large $V$, every prime $p < V$ appears as a hull count at some rotation with rational spread parameters.
\end{conjecture}

The mechanism is \textbf{symmetry reduction via projection}. A polyhedron with $V$ vertices and symmetry group $G$ projects to 2D, breaking $G$ into a subgroup. The hull count $k = V$ minus the number of vertices that become interior points or collinear with hull edges. For a vertex to be ``lost'' from the hull, it must be expressible as a convex combination of its neighbors in the projected plane---an algebraic condition on the spreads. The rotation space $SO(3)$ is therefore partitioned into cells by algebraic surfaces, and each cell has a constant hull count.

The key insight: \textbf{these cell boundaries are algebraic in the spreads}. Rational spread triples $(s_1, s_2, s_3)$ sample the cells at algebraically significant points. The hull counts at those points include primes because the symmetry-breaking removes vertices in ways that do not respect the original $V$'s factorization.

In its most compact form:

\begin{tcolorbox}[colback=blue!5, colframe=blue!40!black, title=\textbf{The Projective Shadow Principle}]
\begin{center}
\emph{Prime polygons are projective shadows of non-prime polyhedra at rational angles.}
\end{center}

\smallskip

This is a \textbf{combinatorial} result about projections, not a \textbf{constructibility} result about regular polygons. It operates in a fundamentally different mathematical category from Gauss-Wantzel.
\end{tcolorbox}

\subsubsection{Observed Vertex-to-Prime Ratios}

The ratio $V/p$ (source vertices to prime hull count) provides an empirical bound:

\begin{center}
\begin{tabular}{cccc}
\toprule
\textbf{Prime $p$} & \textbf{Source} & \textbf{$V$} & \textbf{$V/p$} \\
\midrule
5 & Truncated Tet & 12 & 2.40 \\
7 & Geodesic Tet f=2 & 10 & 1.43 \\
11 & Geodesic Tet f=4 & 34 & 3.09 \\
13 & Geodesic Tet f=4 & 34 & 2.62 \\
\bottomrule
\end{tabular}
\end{center}

The ratios cluster around 1.4--3.1. The geodesic tet f=2 achieves the remarkable ratio $V/p = 10/7 \approx 1.43$---nearly optimal. Whether the lower bound is $V \geq p+3$ (just enough vertices for some to become interior points) or $V \geq 2p$ (requiring substantial reduction) is an open question.

\subsection{Claims of Novelty}

To preempt misunderstanding, we clarify what this work does and does not claim.

\subsubsection{What We Do NOT Claim}

\begin{itemize}
    \item \textbf{We do not ``solve'' the Gauss-Wantzel impossibility.} The heptagon remains non-constructible with compass and straightedge. Our projections do not change this classical result.

    \item \textbf{We do not claim that projection yields a perfectly regular heptagon.} The 7-vertex hull from TruncTet+DualTet at $s=(\frac{1}{2}, \frac{1}{2}, \frac{1}{2})$ achieves regularity 0.861 (angle variance 4.4°, edge variance 8.2\%)---substantially more regular than earlier single-polyhedron results, but still not a perfect regular heptagon. The compound approach and rational spreads yield much better geometry than previous decimal-search results.

    \item \textbf{We do not claim priority on dimensional projection methods.} De Bruijn's 1981 work on Penrose tilings established that ``forbidden'' symmetries can emerge via projection from higher dimensions. Our work extends this insight to prime vertex counts.

    \item \textbf{We do not claim the truncated tetrahedron is newly discovered.} It is a classical Archimedean solid known since antiquity.
\end{itemize}

\subsubsection{What We DO Claim as Novel}

\begin{enumerate}
    \item \textbf{The Symmetry Barrier theorem and its resolution.} We identify why centrally symmetric polytopes \emph{cannot} produce prime projections (all hull counts are even), and demonstrate that asymmetric polytopes like the truncated tetrahedron \emph{can}.

    \item \textbf{Rational-spread parameterization of projection angles.} Prior projection work (e.g., quasicrystals) typically uses irrational ``golden angle'' orientations. We demonstrate that \emph{Tier~1 rational} spread values---specifically $s = (1/2, 1/2, 1/2)$ for the 7-gon and $s = (3/4, 1/4, 1/2)$ for the 11-gon---suffice to produce prime hull counts. These use only denominators $\{2, 4\}$ and radicals $\sqrt{2}, \sqrt{3}$.

    \item \textbf{The 4D± Quadray formulation.} The truncated tetrahedron's vertices are \emph{all rational} in Quadray coordinates: permutations of $(2,1,0,0)/3$. This maintains algebraic exactness throughout the construction, with radicals entering only at the final projection step.

    \item \textbf{Systematic computational search methodology.} We provide reproducible scripts and parameters for discovering prime projections, enabling extension to higher primes (11, 13, ...) as computational resources permit.

    \item \textbf{Interactive visualization in ARTexplorer.} The web-based demonstration allows direct manipulation and verification of the projection phenomenon.

    \item \textbf{Single-polyhedra prime projections (v5.0--5.1).} We demonstrate that prime hulls emerge from a \emph{single polyhedron family}---the geodesic tetrahedron---at frequencies 2 and 4, producing 7-gon, 11-gon, and 13-gon at Tier~1 rational spreads. No compound engineering, no golden ratio. The 13-gon from geodesic tet f=4 at $s=(\frac{1}{2}, \frac{3}{4}, \frac{3}{4})$ beats the compound result in regularity (0.448 vs 0.346), tier (1 vs 3), and degeneracy status (clean vs borderline 179°).

    \item \textbf{Degeneracy verification and the strengthened Symmetry Barrier (v5.1).} We discover that centrally symmetric polyhedra (cuboctahedron, rhombic dodecahedron, geodesic octahedron) produce \emph{exclusively degenerate} odd-gon projections (180° interior angles). This strengthens the Central Symmetry Barrier theorem: not only do these polyhedra tend to produce even hull counts, but when they do produce odd counts, the odd vertices are geometrically meaningless.

    \item \textbf{Path~C exact arithmetic verification (v5.0).} We eliminate all floating-point ambiguity from hull counting via \texttt{Fraction}-based cross products. Every hull count in the single-polyhedra search is provably correct within the Float64 projection model.

    \item \textbf{The Projective Shadow Principle (v5.0).} We formalize the conjecture that prime polygons are projective shadows of non-prime polyhedra at rational angles, with empirical vertex-to-prime ratios suggesting $V \geq 1.5p$ suffices.
\end{enumerate}

\subsubsection{Relationship to Prior Art}

\begin{table}[h]
\centering
\begin{tabular}{lcc}
\toprule
\textbf{Approach} & \textbf{Prior Work} & \textbf{This Work} \\
\midrule
Projection source & 5D hypercubic lattice & 3D/4D polytopes \\
Projection angle & Irrational (golden) & Rational spreads \\
Target symmetry & 5-fold (quasicrystal) & Prime $n$-gon hull \\
Coordinate system & Cartesian & Quadray (WXYZ) \\
Algebraic status & Involves $\phi$ & Purely rational until projection \\
\bottomrule
\end{tabular}
\caption{Comparison with de Bruijn's quasicrystal projection method}
\end{table}

The key novelty is not the \emph{fact} of projection, but the \emph{rationality} of the entire construction chain: rational Quadray vertices $\to$ rational spread rotation $\to$ prime hull count. This aligns with Wildberger's Rational Trigonometry program in a way that irrational-angle projections do not.

\subsection{Quadray Coordinate Formulation}

The discussion above implicitly uses Cartesian $(x, y, z)$ coordinates. However, the \textbf{Quadray coordinate system} $(W, X, Y, Z)$---with basis vectors pointing to tetrahedron vertices---offers significant advantages for rational polygon construction.

\subsubsection{Tetrahedron: Integer Coordinates}

In Quadray, the regular tetrahedron has trivially simple vertices:

\begin{center}
\begin{tabular}{lcc}
\toprule
\textbf{Vertex} & \textbf{Quadray} $(W,X,Y,Z)$ & \textbf{Cartesian} $(x,y,z)$ \\
\midrule
$W$ & $(1, 0, 0, 0)$ & $(1, 1, 1)/\sqrt{3}$ \\
$X$ & $(0, 1, 0, 0)$ & $(1, -1, -1)/\sqrt{3}$ \\
$Y$ & $(0, 0, 1, 0)$ & $(-1, 1, -1)/\sqrt{3}$ \\
$Z$ & $(0, 0, 0, 1)$ & $(-1, -1, 1)/\sqrt{3}$ \\
\bottomrule
\end{tabular}
\end{center}

The Quadray coordinates are \textbf{integer} while Cartesian requires $\sqrt{3}$.

\subsubsection{Truncated Tetrahedron: Rational Coordinates}

The truncated tetrahedron---source of the 7-gon projection---has 12 vertices. In Quadray, \textbf{all coordinates are rational}:

\begin{equation}
\begin{aligned}
\text{Near } W: \quad &(2,1,0,0), \; (2,0,1,0), \; (2,0,0,1) \\
\text{Near } X: \quad &(1,2,0,0), \; (0,2,1,0), \; (0,2,0,1) \\
\text{Near } Y: \quad &(1,0,2,0), \; (0,1,2,0), \; (0,0,2,1) \\
\text{Near } Z: \quad &(1,0,0,2), \; (0,1,0,2), \; (0,0,1,2)
\end{aligned}
\end{equation}

(Normalized by factor of $1/3$.) Compare with the Cartesian form, which requires $\sqrt{2}$ for the edge lengths.

\subsubsection{Basis Vector Spread}

The spread between any two Quadray basis vectors is:
\begin{equation}
s = \sin^2(109.47°) = \frac{8}{9} \quad \text{(exact rational)}
\end{equation}
with $\cos(109.47°) = -\frac{1}{3}$ (also rational). This is the \textbf{natural angle} of tetrahedral geometry---inherently compatible with Rational Trigonometry.

\subsubsection{Rotation Coefficients}

For a rotation by spread $s$ in the Quadray system, we define:
\begin{equation}
F = \frac{2\cos\theta + 1}{3}, \quad G = \frac{1 - \cos\theta + \sqrt{3}\sin\theta}{3}, \quad H = \frac{1 - \cos\theta - \sqrt{3}\sin\theta}{3}
\end{equation}

where $\sin\theta = \sqrt{s}$ and $\cos\theta = \sqrt{1-s}$. For rational spreads, the rotation remains algebraically exact until the final $\sqrt{\cdot}$ operation.

\subsubsection{Advantage Summary}

\begin{center}
\begin{tabular}{lcc}
\toprule
\textbf{Property} & \textbf{Quadray} $(W,X,Y,Z)$ & \textbf{Cartesian} $(x,y,z)$ \\
\midrule
Tetrahedron vertices & Integer $(1,0,0,0)$ & Irrational $(\sqrt{3})$ \\
Truncated tetrahedron & Rational $(2,1,0,0)/3$ & Irrational $(\sqrt{2})$ \\
Basis vector angle & Spread $8/9$ (rational) & $\cos^{-1}(-1/3)$ (transcendental) \\
IVM lattice & Native & Requires conversion \\
\bottomrule
\end{tabular}
\end{center}

The Quadray formulation maintains rationality longer than Cartesian, deferring radical evaluation until the final projection step. This makes it the natural coordinate system for Rational Synergetics and prime polygon projections.

%==============================================================================
\section{Future Work}
%==============================================================================

\subsection{Higher Primes (17, 19, 23...)}

With all four initial primes now verified at rational spreads (5, 7, 11, 13), the next targets are:

\begin{enumerate}
    \item \textbf{17-gon (heptadecagon)}: Gauss-constructible but requires degree-16 polynomial; ironically easier to construct classically than to project! The geodesic approach (higher frequency subdivision) may provide sufficient vertices.
    \item \textbf{19-gon}: Likely needs $V \geq 29$ vertices (based on the $V/p \geq 1.5$ empirical bound). Geodesic icosahedron at frequency~2 (42 vertices) is a strong candidate.
    \item \textbf{23-gon}: Requires $V \geq 35$. Higher-frequency geodesic polyhedra or compound of three asymmetric polyhedra.
\end{enumerate}

\textbf{Scaling strategies} (v5.0): Two complementary approaches:
\begin{itemize}
    \item \textbf{Geodesic subdivision}: Increase Fuller frequency to generate arbitrarily many vertices from a single polyhedron. Geodesic icosahedron at $f=3$ yields 92 vertices---likely sufficient for primes up to $\sim$60.
    \item \textbf{Compound polyhedra}: Each additional asymmetric polyhedron adds vertices and breaks further symmetry. A compound of truncated tetrahedron + icosahedron + dodecahedron (44 vertices) might yield 17-gon or 19-gon.
\end{itemize}

\subsection{Geodesic Frequency Scaling}

The single-polyhedra results (v5.0) open a systematic approach: for a base polyhedron with $F$ triangular faces, geodesic subdivision at frequency $f$ produces approximately $F \cdot f^2 / 2$ vertices. This means:

\begin{center}
\begin{tabular}{lcccc}
\toprule
\textbf{Base} & \textbf{Freq} & \textbf{Vertices} & \textbf{Max prime (est.)} & \textbf{Confirmed} \\
\midrule
Tetrahedron & $f=2$ & 10 & 7 & \textbf{7} \\
Tetrahedron & $f=3$ & 20 & 13 & 11 \\
Tetrahedron & $f=4$ & 34 & 23 & \textbf{11, 13} \\
Tetrahedron & $f=5$ & 52 & $\sim$35 & --- \\
Octahedron & $f=2$ & 18 & 11 & \textcolor{red}{degenerate} \\
Octahedron & $f=3$ & 38 & 23 & --- \\
Icosahedron & $f=2$ & 42 & 27 & --- \\
Icosahedron & $f=3$ & 92 & $\sim$60 & --- \\
\bottomrule
\end{tabular}
\end{center}

\textbf{Key finding (v5.1):} Only the geodesic \emph{tetrahedron} produces clean (non-degenerate) prime hulls. The geodesic octahedron (centrally symmetric) produces exclusively degenerate odd-gon results. The geodesic icosahedron (also centrally symmetric) is predicted to behave similarly. An open question: \textbf{does the geodesic tetrahedron at frequency $f$ produce at least one prime hull count for all $f \geq 2$?} The confirmed results ($f=2 \to 7$, $f=3 \to 11$, $f=4 \to 11,13$) suggest a systematic scaling.

\subsection{Classifying Algebraic Cell Structure}

The Projective Shadow Principle implies that $SO(3)$ is partitioned into cells by algebraic surfaces, with each cell having a constant hull count. Future work should:
\begin{itemize}
    \item Characterize the \textbf{algebraic degree} of the cell boundaries as a function of $V$
    \item Determine whether Tier~1 rational spreads always intersect a prime-count cell
    \item Prove or disprove the conjectured bound $V \geq 1.5p$
\end{itemize}

\subsection{Janus Polarity Perturbation}

The 4D± system includes Janus polarity---a discrete $\pm$ state. Applying opposite polarities to paired vertices might break symmetry in centrally symmetric polytopes, potentially enabling prime projections from regular polytopes.

\subsection{Symbolic Verification}

Path~C exact arithmetic (Section~\ref{sec:pathc}) proves hull counts are deterministic, but a stronger result would be a \textbf{closed-form proof}:
\begin{itemize}
    \item Derive exact algebraic expressions for the hull vertices
    \item Verify that exactly $n$ vertices lie on the boundary for prime $n$
    \item Express the viewing transformation in closed form
    \item Determine whether the hull count is invariant under perturbation of the spread within its algebraic cell
\end{itemize}

\subsection{ARTexplorer Interactive Demo}

The prime projection discovery is implemented as an interactive demonstration in the ARTexplorer web application at \url{https://arossti.github.io/ARTexplorer/}.

\subsubsection{Accessing the Demo}

\begin{enumerate}
    \item Open the \textbf{Math Demos} menu in the application toolbar
    \item Select \textbf{Prime Projections} to open the floating panel
    \item Click any preset button: \textbf{5-gon}, \textbf{7-gon}, \textbf{11-gon}, \textbf{13-gon} (compound, red) or \textbf{7\bigstar}, \textbf{11\bigstar}, \textbf{13\bigstar} (single-poly geodesic tet, blue)
    \item Each button enables the correct polyhedron and applies the rational spread preset
\end{enumerate}

The yellow hull / cyan ideal polygon / green ray visualization shows the prime n-gon that emerges from the polyhedron's projection at the rational spread viewing angle. Each preset's rational spreads are stored in \texttt{PROJECTION\_PRESETS} in \texttt{rt-prime-cuts.js}. Blue-starred buttons (\bigstar) indicate single-polyhedra geodesic tet results.

\begin{figure}[p]
    \centering
    \includegraphics[width=0.85\linewidth]{5-gon.png}
    \vspace{0.3cm}
    \caption{5-gon projection: Truncated Tetrahedron at rational spread $s=(0,\frac{1}{2},0)$. The pentagon emerges as the convex hull boundary of the projected vertices.}
    \label{fig:5gon-demo}

    \vspace{0.8cm}

    \includegraphics[width=0.85\linewidth]{7-gon.png}
    \vspace{0.3cm}
    \caption{7-gon projection: TruncTet+DualTet compound at $s=(\frac{1}{2},\frac{1}{2},\frac{1}{2})$---all three rational spreads equal. Tier~1, regularity 0.861.}
    \label{fig:7gon-demo}
\end{figure}

\subsubsection{Condensed Quadray Formulations}

The UI displays condensed formulas explaining each projection's construction:

\begin{tcolorbox}[colback=cyan!5, colframe=cyan!40!black, title=\textbf{7-gon Projection Formula (v4.0)}]
\begin{verbatim}
7-gon: TruncTet+DualTet (16v) → s=(½, ½, ½)
Tier 1 rational, regularity 0.861
\end{verbatim}

\textbf{Interpretation:}
\begin{itemize}
    \item \textbf{Source}: Truncated tetrahedron (12v) + dual tetrahedron (4v), unit-sphere normalized
    \item \textbf{Viewing spreads}: $s_1 = s_2 = s_3 = 1/2$---\emph{all spreads identical and rational}
    \item \textbf{Rotation matrix}: Pure $\sqrt{2}/2$ entries, cached in \texttt{RT.PureRadicals}
    \item \textbf{Hull reduction}: 16 vertices $\to$ 7-vertex boundary (9 project to interior)
    \item \textbf{Tier}: 1 (RT-pure)---denominators $\{2\}$ only
\end{itemize}
\end{tcolorbox}

\begin{tcolorbox}[colback=cyan!5, colframe=cyan!40!black, title=\textbf{11-gon Projection Formula (v4.0)}]
\begin{verbatim}
11-gon: TruncTet+Icosa (24v) → s=(¾, ¼, ½)
Tier 1 rational, regularity 0.490
\end{verbatim}

\textbf{Interpretation:}
\begin{itemize}
    \item \textbf{Source}: Truncated tetrahedron (12v) + icosahedron (12v), unit-sphere normalized
    \item \textbf{Viewing spreads}: $s = (3/4, 1/4, 1/2)$---denominators $\{2, 4\}$
    \item \textbf{Rotation entries}: $\sqrt{1/4} = 1/2$, $\sqrt{3/4} = \sqrt{3}/2$, $\sqrt{1/2} = \sqrt{2}/2$
    \item \textbf{Significance}: Non-constructible 11-gon at the \emph{same radical family} as 5-gon and 7-gon
\end{itemize}
\end{tcolorbox}

\subsubsection{Technical Implementation}

The projection overlay is implemented in \texttt{modules/rt-projections.js} and \texttt{modules/rt-prime-cuts.js}:
\begin{itemize}
    \item \texttt{PROJECTION\_PRESETS} registry in \texttt{rt-prime-cuts.js}---single source of truth for all 7 presets (4 compound + 3 geodesic tet)
    \item Each preset stores \texttt{rationalSpreads} (e.g.\ \texttt{["1/2", "1/2", "1/2"]}) and \texttt{rationalTier}
    \item Yellow hull / cyan ideal polygon / green projection rays visualization
    \item Floating modal panel accessible from Math Demos menu
\end{itemize}

Rational spread presets are applied via \texttt{RTProjections.applyPreset()}, which constructs the rotation matrix from the stored spread values.

%==============================================================================
\section{Conclusion}
%==============================================================================

We have demonstrated that:

\begin{enumerate}
    \item \textbf{The Symmetry Barrier exists}: Regular polytopes with central symmetry always project to even-sided hull boundaries, precluding prime n-gons (except 2).

    \item \textbf{The Symmetry Barrier can be broken}: Asymmetric polytopes (truncated tetrahedron), compound polyhedra, and even centrally symmetric solids at non-axial viewing angles produce odd hull counts including prime counts.

    \item \textbf{Prime projections exist at exact rational spreads for 5, 7, 11, and 13}:
    \begin{itemize}
        \item \textbf{5-gon}: Truncated tetrahedron at $s=(0, \frac{1}{2}, 0)$ --- Tier~1
        \item \textbf{7-gon}: TruncTet+DualTet at $s=(\frac{1}{2}, \frac{1}{2}, \frac{1}{2})$ --- Tier~1, all spreads equal
        \item \textbf{11-gon}: TruncTet+Icosa at $s=(\frac{3}{4}, \frac{1}{4}, \frac{1}{2})$ --- Tier~1
        \item \textbf{13-gon}: TruncTet+Icosa at $s=(\frac{9}{10}, \frac{24}{25}, \frac{19}{20})$ --- Tier~3
    \end{itemize}

    \item \textbf{A single polyhedron family produces all three non-constructible primes} \textcolor{green!50!black}{[REVISED v5.1]}:
    \begin{itemize}
        \item \textbf{7-gon}: Geodesic Tet f=2 (10v) at $s=(0, \frac{1}{3}, \frac{1}{3})$ --- Tier~1, $\sqrt{2}/\sqrt{3}$
        \item \textbf{11-gon}: Geodesic Tet f=4 (34v) at $s=(\frac{3}{4}, \frac{1}{3}, \frac{1}{3})$ --- Tier~1, $\sqrt{2}/\sqrt{3}$
        \item \textbf{13-gon}: Geodesic Tet f=4 (34v) at $s=(\frac{1}{2}, \frac{3}{4}, \frac{3}{4})$ --- Tier~1, $\sqrt{2}$ only
    \end{itemize}

    \item \textbf{The Central Symmetry Barrier is stronger than expected} \textcolor{green!50!black}{[NEW v5.1]}: Centrally symmetric polyhedra not only tend to produce even hull counts---when they do produce odd counts, those results are \emph{exclusively degenerate} (180° interior angles). Only asymmetric polyhedra (geodesic tetrahedron) yield clean odd hulls.

    \item \textbf{All four primes now at Tier~1} \textcolor{green!50!black}{[REVISED v5.1]}: With the geodesic tet f=4's 13-gon at $s=(\frac{1}{2}, \frac{3}{4}, \frac{3}{4})$, all four primes (5, 7, 11, 13) emerge at Tier~1 denominators $\{2, 3, 4\}$, radicals $\sqrt{2}$ and $\sqrt{3}$. The Tier~3 compound 13-gon is superseded.

    \item \textbf{Exact arithmetic eliminates floating-point ambiguity} \textcolor{green!50!black}{[v5.0]}: Path~C \texttt{Fraction}-based hull computation proves that all reported hull counts are deterministic and correct within the Float64 projection model.

    \item \textbf{The Projective Shadow Principle} \textcolor{green!50!black}{[v5.0]}: We conjecture that \emph{prime polygons are projective shadows of non-prime polyhedra at rational angles}---a combinatorial result about projection geometry that is fundamentally distinct from compass-and-straightedge constructibility.
\end{enumerate}

The most striking result is the \textbf{geodesic tetrahedron's universality}: a single polyhedron family, at increasing subdivision frequency, produces \emph{all three non-constructible primes} (7, 11, 13) at Tier~1 rational spreads. No golden ratio, no compound engineering, no transcendental functions. The geodesic tetrahedron's 13-gon at $s=(\frac{1}{2}, \frac{3}{4}, \frac{3}{4})$ uses only $\sqrt{2}$ radicals and beats the compound in regularity, tier, and degeneracy status. The prime emerges from the geometry of projection itself---from the tetrahedron, the simplest possible polyhedron, subdivided by Fuller's geodesic method.

This work connects four mathematical traditions:
\begin{itemize}
    \item \textbf{Gauss-Wantzel constructibility}---the classical limits of compass and straightedge
    \item \textbf{Wildberger's Rational Trigonometry}---avoiding transcendentals through spread/quadrance
    \item \textbf{Fuller's Synergetics}---tetrahedral geometry, geodesic subdivision, the Vector Equilibrium
    \item \textbf{Projection from higher dimensions}---the quasicrystal insight that ``impossible'' symmetries can be shadows of higher-dimensional regularity
\end{itemize}

The prime n-gon, forbidden by Gauss and Wantzel in 2D, is found as a shadow cast from the geometry of Fuller's synergetics, viewed through the lens of Wildberger's rational trigonometry.

%==============================================================================
\section*{Acknowledgments}
%==============================================================================

This work draws on:
\begin{itemize}
    \item N.J.\ Wildberger's Rational Trigonometry
    \item R.\ Buckminster Fuller's Synergetic Geometry
    \item Kirby Urner's Quadray coordinate system
    \item de Bruijn's projection methods for Penrose tilings
\end{itemize}

Developed in collaboration with AI assistance (Claude/Anthropic) for algorithm design, mathematical formalization, and documentation.

%==============================================================================
\section*{References}
%==============================================================================

\begin{enumerate}
    \item Wildberger, N.J. (2005). \emph{Divine Proportions: Rational Trigonometry to Universal Geometry}. Wild Egg Books.
    \item Gauss, C.F. (1801). \emph{Disquisitiones Arithmeticae}. Leipzig.
    \item Wantzel, P.L. (1837). ``Recherches sur les moyens de reconna\^itre si un Probl\`eme de G\'eom\'etrie peut se r\'esoudre avec la r\`egle et le compas.'' \emph{Journal de Math\'ematiques Pures et Appliqu\'ees}.
    \item de Bruijn, N.G. (1981). ``Algebraic theory of Penrose's non-periodic tilings of the plane.'' \emph{Indagationes Mathematicae}.
    \item Fuller, R.B. (1975). \emph{Synergetics: Explorations in the Geometry of Thinking}. Macmillan.
    \item Thomson, A. (2026). ``Spread-Quadray Rotors: A Tetrahedral Alternative to Quaternions.'' Open Building / ARTexplorer Project.
    \item Thomson, A. (2026). ``Geometric Janus Inversion: Extending the Janus Point from Temporal to Spatial Geometry via Tetrahedral (Quadray) Coordinates.'' Open Building / ARTexplorer Project. (Companion paper on 4D$^\pm$ Quadray space and zero-sum normalization.)
    \item Thomson, A. (2026). \emph{ARTexplorer: Interactive Tetrahedral Geometry Visualization}. \url{https://arossti.github.io/ARTexplorer/}
\end{enumerate}

\vfill

\noindent\rule{\textwidth}{0.4pt}
\smallskip
\noindent DOI: 10.13140/RG.2.2.23043.98089 | CC BY-NC-ND 4.0 | \copyright\ Andy Ross Thomson, M.Arch, OAA 2026 | February 2026

\begin{center}
\textit{``Prime polygons are projective shadows of non-prime polyhedra\\
at rational angles---forbidden by Euclid's tools,\\
yet found in the geometry of Fuller's light.''}
\end{center}

\vspace{0.5cm}
\noindent\textbf{Version History:}
\begin{itemize}[noitemsep]
    \item v1.0 (Jan 2026): Initial conjecture and symmetry barrier discovery
    \item v2.0 (Feb 2026): 5-gon and 7-gon projections verified
    \item v3.0 (Feb 2026): 11-gon and 13-gon breakthrough via compound polyhedra
    \item v3.1 (Feb 2026): Project-Streamline---unified Python/JS definitions, verified spreads
    \item v4.0 (Feb 2026): Path A---all 4 primes at exact rational spreads; 5/7/11 at Tier~1
    \item v5.0 (Feb 2026): Single-polyhedra prime projections; Path~C exact arithmetic; geodesic subdivision; Projective Shadow Principle conjecture
    \item \textbf{v5.1 (Feb 2026): Degeneracy verification reveals central symmetry barrier is stronger---cuboctahedron, rhombic dodec, geodesic oct results all degenerate (180°). Geodesic tet f=4 produces first single-poly 13-gon (Tier~1, $\sqrt{2}$ only). All four primes now at Tier~1 from single polyhedra.}
\end{itemize}

\end{document}
